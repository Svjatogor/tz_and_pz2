\newcommand{\CommonSoftwareRequirements}{
\begin{itemize}
\item git;
\item nvm;
\item Node.js~v6.0.0;
\item npm~v3.8.6;
\item MongoDB~v2.6.12;
\item Stardog~v4.1.
\end{itemize}
}

\newcommand{\CommonBrowserRequirements}{
\begin{itemize}
\item Google Chrome~30.0;
\item Mozilla Firefox~25.0;
\item Apple Safari~6;
\item Microsoft Internet Explorer~9.
\end{itemize}
}

\newcommand{\CommonSystemRequirements}{
\begin{itemize}
\item процессор с тактовой частотой не ниже 2.7 ГГц;
\item объем оперативной памяти не ниже 512 Мб;
\item объем жесткого диска не ниже 10 Гб.
\end{itemize}
}

\newcommand{\CommonClientRequirements}{
\begin{itemize}
\item процессор с тактовой частотой не ниже 1600 МГц;
\item объем оперативной памяти не ниже 1024 Мб;
\item объем жесткого диска не ниже 20 Гб;
\item монитор с диагональю не менее 15 дюймов;
\item манипулятор типа «мышь», клавиатура.
\end{itemize}
}

\newcommand{\CommonGoal}{повышение качества принимаемых решений по управлению отходами на предприятии за счет организации интеллектуальной поддержки процесса принятия решений на основе онтологической модели представления знаний и логического вывода на онтологии}


\newcommand{\CommonGoalCap}{Повышение качества принимаемых решений по управлению отходами на предприятии за счет организации интеллектуальной поддержки процесса принятия решений на основе онтологической модели представления знаний и логического вывода на онтологии}

\newcommand{\CommonTasks}{
\begin{itemize}
\item провести анализ процесса и специфики управления отходами на предприятии с целью построения информационно-логической модели предметной области и формирования требований к модели представления знаний;
\item разработать концепцию поддержки принятия решений по управлению отходами предприятия на основе онтологической модели представления знаний и логического вывода на онтологии;
\item разработать онтологическую модель предметной области и алгоритм генерации стратегии управления отходами предприятия на основе логического вывода на онтологии;
\item разработать и протестировать интеллектуальную систему поддержки принятия решений по управлению отходами на основе предложенных модели и алгоритма.
\end{itemize}
}

\def\CommonOntologyModelFormula{O = \left<C,~R,~F\right >}
\def\CommonOntologyModelDescription{\begin{VSTUFormulaWhereList}
\item $C$~--- конечное множество понятий (концептов) предметной области, которую представляет онтология О;
\item $R$~--- конечное множество отношений между концептами;
\item $F$~--- конечное множество функций интерпретации, заданных на концептах и/или отношениях онтологии О.
\end{VSTUFormulaWhereList}
}

\newcommand{\CommonTprFormulaCompact}{$DS = \left<S,~A,~C,~M,~P,~R\right>$, где $S$~-- описание ситуации принятия решений, состоящее из множества численных и качественных параметров: $S = \left\{P_{q1},~P_{q2},~...,~P_{qn1},~P_{qn2}\right\}$; $A$~-- множество альтернатив, каждая из которых состоит из множества управляющих воздействий: $A = \left\{A_{c1},~A_{c2},~...,\right\}$; $C$~-- множество критериев, в виде качественных оценок ситуации с точки зрения предприятия; $M$~-- модель, позволяющая для каждой альтернативы рассчитать вектор критериев; $P$~-- система предпочтений для каждого из критериев; $R$~-- решающее правило выбора альтернативы}

\def\CommonTprFormula{DS = \left<S,~A,~C,~M,~P,~R\right>}
\def\CommonTprDescription{\begin{VSTUFormulaWhereList}
\item $S$~--- описание ситуации принятия решений, состоящее из множества численных и качественных параметров, $S = \left\{P_{q1},~P_{q2},~...,~P_{qn1},~P_{qn2}~\right\}$;
\item $A$~--- множество альтернатив, каждая из которых состоит из множества управляющих воздействий: $A = \left\{A_{c1},~A_{c2},~...,~A_{cm}\right\}$;
\item $C$~--- множество критериев, в виде качественных оценок ситуации с точки зрения предприятия;
\item $M$~--- модель, позволяющая для каждой альтернативы рассчитать вектор критериев;
\item $P$~--- система предпочтений для каждого из критериев;
\item $R$~--- решающее правило выбора альтернативы.
\end{VSTUFormulaWhereList}
}

\newcommand{\CommonMetaontologyModels}{
\begin{itemize}
\item онтология отходов, описывающая их свойства и классы опасности, а также негативное влияние, которые они оказывают на окружающую среду;
\item онтология субъектов, взаимодействующих с отходами (предприятие, полигон и т. д.);
\item онтология методов управления отходами, описывающая методы (переработка, утилизации, транспортировки и т.д.) и их стоимость, экологический вред, который также должен оплачиваться субъектом согласно закону РФ.
\end{itemize}
}

\newcommand{\CommonMetaontologyFormulaCompact}{$M = \left<O_{M},~C_{M},~Inst_{M},~R_{M},~I_{M}\right>$, 

где $M$ -- метаонтологическая модель предметной области; $O_{M} = \left\{O_{W},~O_{M},~O_{S}\right\}$~-- множество онтологических моделей, объединенных в метаонтологию, $O_{W}$~-- онтология отходов, $O_{M}$~-- онтология методов управления отходами,  $O_{S}$~-- онтология субъектов управления отходами; $C_{M}$~-- конечное множество концептов, $C_{M} = \varnothing$; $Inst_{M}$~-- конечное множество экземпляров классов, $Inst_{M} = \varnothing$; $R_{M} = \left\{has,~uses,~includes\right\}$~-- конечное множество отношений метаонтологии; $I_{M}$~-- множество правил интерпретации и ограничений, $I_{M} = \varnothing$}

\def\CommonMetaontologyFormula{M = \left<O_{M},~C_{M},~Inst_{M},~R_{M},~I_{M}\right>}
\def\CommonMetaontologyDescription{\begin{VSTUFormulaWhereList}
\item $M$~--- метаонтологическая модель предметной области;
\item $O_{M}$~--- множество онтологических моделей, объединенных в метаонтологию: $O_{M} = \left\{O_{W},~O_{M},~O_{S}\right\}$, где $O_{W}$~-- онтология отходов, $O_{M}$~-- онтология методов управления отходами,  $O_{S}$~-- онтология субъектов управления отходами;
\item $C_{M}$~--- конечное множество концептов, $C_{M} = \varnothing$;
\item $Inst_{M}$~--- конечное множество экземпляров классов, $Inst_{M} = \varnothing$;
\item $R_{M}$~--- конечное множество отношений метаонтологии $M$: $R = \left\{r_{M1},~r_{M2},~r_{M3}\right\}$, где $r_{M1}$~-- отношение has «имеет», $r_{M2}$~-- отношение uses «использует», $r_{M3}$~-- отношение includes «включает»;
\item $I_{M}$~--- множество правил интерпретации и ограничений, $I_{M} = \varnothing$.
\end{VSTUFormulaWhereList}
}

\newcommand{\CommonWasteOntologyCompact}{$O_{W} = \left<C_{W},~Inst_{W},~R_{W},~I_{W}\right>$, 

где $C_{W}$~-- конечное множество концептов онтологии отходов, $C_{W} = \left\{C_{W1},~...,~C_{W26}\right\}$; $Inst_{W}$~-- множество экземпляров классов онтологии отходов, $Inst_{W} = \left\{i_{W1},~i_{W2},~...,~i_{Wj},~...,~i_{Wn}\right\}$; $R_{W}$~-- конечное множество отношений онтологии отходов: $R_{W} = \left\{r_{W1},~...,~r_{W8}\right\}$, где $r_{W1}$~-- отношение $hasHazard$, $r_{W2}$~-- отношение $hasAggregateState$, $r_{W3}$~-- relation $hasOrigin$, $r_{W4}$~-- отношение $hasAmount$, $r_{W5}$~-- отношение $hasTitle$, $r_{W6}$~-- отношение $hasEcolTax$, $r_{W7}$~-- отношение $is$, $r_{W8}$~-- отношение $is-a$; $I_{W}$~-- множество правил}

\def\CommonWasteOntologyFormula{O_{W} = \left<C_{W},~Inst_{W},~R_{W},~I_{W}\right>}
\def\CommonWasteOntologyDescription{\begin{VSTUFormulaWhereList}
\item $C_{W}$~--- конечное множество концептов онтологии отходов: $C_{W} = \left\{C_{W1},~...,~C_{W26}\right\}$;
\item $Inst_{W}$~--- множество экземпляров классов онтологии отходов: $Inst_{W} = \left\{i_{W1},~i_{W2},~...,~i_{Wj},~...,~i_{Wn}\right\}$;
\item $R_{W}$~--- конечное множество отношений онтологии отходов: $R_{W} = \left\{r_{W1},~...,~r_{W8}\right \}$, где $r_{W1}$~-- отношение $hasHazard$, $r_{W2}$~-- отношение $hasAggregateState$, $r_{W3}$~-- relation $hasOrigin$, $r_{W4}$~-- отношение $hasAmount$, $r_{W5}$~-- отношение $hasTitle$, $r_{W6}$~-- отношение $hasEcolTax$, $r_{W7}$~-- отношение $is$, $r_{W8}$~-- отношение $is-a$;
\item $I_{W}$~--- множество правил и ограничений (методика их добавления рассмотрена далее).
\end{VSTUFormulaWhereList}
}

\newcommand{\CommonMethodOntologyCompact}{$O_{M} = \left<C_{M},~Inst_{M},~R_{M},~I_{M}~\right>$, где $C_{M}$~-- конечное множество концептов онтологии методов управления отходами, $C_{M} = \left\{C_{M1},~...,~C_{M6}\right\}$; $Inst_{M}$~-- множество экземпляров классов онтологии методов управления отходами, $Inst_{M} = \left\{i_{M1},~i_{M2},~...,~i_{Mj},~...,~i_{Mn}\right\}$; $R_{M}$~-- множество отношений онтологии методов управления отходами, $R_{M} = \left\{r_{M1},~...,~r_{M8}\right\}$, где $r_{M1}$~-- отношение $is$, $r_{M2}$~-- отношение $hasTitle$, $r_{M3}$~-- отношение $hasMethod$, $r_{M4}$~-- отношение $processedBy$, $r_{M5}$~-- отношение $hasHarmfulEffect$, $r_{M6}$~-- отношение $hasCostOnDistance$, $r_{M7}$~-- отношение $hasCostOnWeight$, $r_{M8}$~-- отношение $hasCostByService$; $I_{M}$~-- множество правил интерпретации и ограничений, $I_{M} = \varnothing$}

\def\CommonMethodOntologyFormula{O_{M} = \left<C_{M},~Inst_{M},~R_{M},~I_{M}\right>}
\def\CommonMethodOntologyDescription{\begin{VSTUFormulaWhereList}
\item $C_{M}$~--- конечное множество концептов онтологии методов управления отходами: $C_{M} = \left\{C_{M1},~...,~C_{M6}\right\}$;
\item $Inst_{M}$~--- множество экземпляров классов онтологии методов управления отходами: $Inst_{M} = \left\{i_{M1},~i_{M2},~...,~i_{Mj},~...,~i_{Mn}\right\}$;
\item $R_{M}$~--- множество отношений онтологии методов управления отходами: $R_{M} = \left\{r_{M1},~...,~r_{M8}\right\}$, где $r_{M1}$~-- отношение $is$, $r_{M2}$~-- отношение $hasTitle$, $r_{M3}$~-- отношение $hasMethod$, $r_{M4}$~-- отношение $processedBy$, $r_{M5}$~-- отношение $hasHarmfulEffect$, $r_{M6}$~-- отношение $hasCostOnDistance$, $r_{M7}$~-- отношение $hasCostOnWeight$, $r_{M8}$~-- отношение $hasCostByService$;
\item $I_{M}$~--- множество правил интерпретации и ограничений, $I_{M} = \varnothing$.
\end{VSTUFormulaWhereList}
}

\newcommand{\CommonSubjectOntologyCompact}{$O_{S} = \left<C_{S},~Inst_{S},~R_{S},~I_{S}\right>$, где $C_{S}$~-- конечное множество концептов субъектов управления отходами, $C_{S} = \left\{C_{S1},~...,~C_{S6}\right\}$; $Inst_{S}$~-- множество экземпляров классов онтологии субъектов управления отходами, $Inst_{S} = \left\{i_{S1},~i_{S2},~...,~i_{Sj},~...,~i_{Sn}\right\}$; $R_{S}$~-- множество отношений онтологии субъектов управления отходами, $R_{S} = \left\{r_{S1},~...,~r_{S9}\right\}$, $r_{S1}$~-- отношение is, $r_{S2}$~-- отношение locatedIn, $r_{S3}$~-- отношение hasCoordinates, $r_{S4}$~-- отношение hasEcolOfGround, $r_{S5}$~-- отношение hasTitle, $r_{S6}$~-- отношение hasWaste, $r_{S7}$~-- отношение hasMethod, $r_{S8}$~-- отношение hasBudget, $r_{S9}$~-- отношение hasEcolOfAir; $I_{S}$~-- множество правил интерпретации и ограничений, $I_{S} = \varnothing$}

\def\CommonSubjectOntologyFormula{O_{S} = \left<C_{S},~Inst_{S},~R_{S},~I_{S}\right>}
\def\CommonSubjectOntologyDescription{\begin{VSTUFormulaWhereList}
\item $C_{S}$~--- конечное множество концептов субъектов управления отходами: $C_{S} = \left\{C_{S1},~...,~C_{S6}\right\}$;
\item $Inst_{S}$~--- множество экземпляров классов онтологии субъектов управления отходами: $Inst_{S} = \left\{i_{S1},~i_{S2},~...,~i_{Sj},~...,~i_{Sn}\right\}$;
\item $R_{S}$~--- множество отношений онтологии субъектов управления отходами: $R_{S} = \left\{r_{S1},~...,~r_{S9}\right\}$, где $r_{S1}$~-- отношение is, $r_{S2}$~-- отношение locatedIn, $r_{S3}$~-- отношение hasCoordinates, $r_{S4}$~-- отношение hasEcolOfGround, $r_{S5}$~-- отношение hasTitle, $r_{S6}$~-- отношение hasWaste, $r_{S7}$~-- отношение hasMethod, $r_{S8}$~-- отношение hasBudget, $r_{S9}$~-- отношение hasEcolOfAir;
\item $I_{S}$~--- множество правил интерпретации и ограничений, $I_{S} = \varnothing$.
\end{VSTUFormulaWhereList}
}

\def\CommonTaskGivenFormula{s = \left<Coord_{s},~L_{s},~B_{s},~W_{s},~M_{s}\right>}
\def\CommonTaskGivenDescription{\begin{VSTUFormulaWhereList}
\item $s$~--- предприятие, являющиеся экземпляром класа $Subject$ онтологии $O_{S}$;
\item $Coord_{s}$~--- координаты предприятия, строка;
\item $L_{s}$~--- локация предприятия, определяющая город, регион и др.: $L_{s} = \left\{l_{1},~l_{2},~...,~l_{i},~...,~l_{n}\right\}$, где $l_{i}$~-- экземпляр класса $Subject$ онтологии $O_{S}$;
\item $B_{s}$~--- бюджет предприятия, число;
\item $W_{s}$~--- отходы предприятия, сгрупированные по методу управления отходами $W_{s} = \left\{W_{1},~W_{2},~...,~W_{i},~...,~W_{n}\right\}$, где $W_{i}$~-- подмножество отходов, объединенных общим признаком -- метод управления $m_{i}$, $W_{i} = \left\{w_{i1},~w_{i2},~...,~w_{ij}\right\}$, где $w_{ij}$~-- экземпляр класса $Waste$ онтологии $O_{W}$;
\item $M_{s}$~--- методы управления отходами предприятия: $M_{s} = \left\{m_{1},~m_{2},~...,~m_{i},~...,~m_{n}\right\}$, где $m_{i}$~--- экземпляр класса $Method$ онтологии $O_{M}$. 
\end{VSTUFormulaWhereList}
}

\def\CommonFormalOntologyTask{M|_{min(ecol, econ)} = \{M_{1},~M_{2}~...,~M_{k}\} : k = |W_{s}|.}
\def\CommonFormalOntologySolution{St_{s} = \{\left<W_{1},~M_{1}\right>,~\left<W_{2},~M_{2}\right>,~...,~\left<W_{n},~M_{n}\right>\}.}

\newcommand{\CommonScientificNovations}{
\begin{itemize}
\item Разработана интегрированная онтологическая модель представления знаний по управлению отходами предприятия, которая отличается от известных возможностью описания данных и знаний об объектах и субъектах процесса управления отходами на общем домене концептов, а также позволяет реализовывать логический вывод на онтологии на основе семантических запросов.
\item Разработан алгоритм генерации эффективной стратегии управления отходами на основе логического вывода на онтологической модели с использованием семантических запросов.
\end{itemize}
}

\newcommand{\CommonPracticalValue}{
\begin{itemize}
\item Разработанные в диссертационной работе модели и алгоритм позволяют производить генерацию эффективной стратегии управления отходами на предприятии. Реализованная система поддержки принятия решений включает в себя модуль генерации стратегии управления отходами на предприятии, а также онтологическую базу знаний отходов, методов и субъектов управления отходами. Разработана методика создания и расширения онтологической базы знаний предметной области, что позволяет применять данную систему, учитывая особенности различных субъектов управления отходами и видов отходов. В результате повышается качество принимаемых решений в области обращения с отходами на предприятии при использовании предложенной системой стратегии управления отходами.
\item Реализованная система поддержки принятия решений прошла аппробацию в учреждении ГБУЗ «Николаевское ЦРБ» в процессе обращения с отходами. 
\item Работа выполнена при поддержке РФФИ в рамках проекта №15-07-03541 «Интеллектуальная поддержка принятия решений по управлению сложными системами на основе интеграции различных типов рассуждений на знаниях, представленных онтологической моделью».
\end{itemize}
}

\newcommand{\CommonResults}{
\begin{itemize}
\item Проведен анализ процессов управления отходами на предприятии; информационных систем, используемых при принятии решений по управлению отходами; обзор моделей и методов, используемых при поддержке принятия решений по управлению отходами. Основным недостатком существующего процесса является сложность и трудоемкость процесса принятия верного решения экпертом в области управления отходами на предприятии. Построена информационно-логическая модель предметной области, включающая функциональную и объектную модели. Выявлены требования к модели представления знаний для описания объектов и субъектов процесса управления отходами на предприятии.
\item Разработана концепция поиска эффективной стратегии управления отходами на предприятии. Предложенная концепция предполагает создание автоматизированной системы для решения задачи генерации стратегии управления отходами, что позволяет сократить трудоемкость решения задачи и повысить обоснованность принимаемых решений, за счет применения моделей и методов искусственного интеллекта -- онтологической модели представления знаний и логического вывода на онтологиях.
\item Разработана интегрированная онтологическая модель представления знаний предметной области, состоящая из следующих компонентов, объединенных метаонтологией: (1) онтология отходов; (2) онтология методов управления отходами; (3) онтология субъектов управления отходами. Разработанная модель позволяет описывать объекты и субъекты процесса управления отходами на общем домене концептов и решать задачу поиска эффективной стратегии управления отходами посредством логического вывода на онтологии. Разработан алгоритм генерации эффективной стратегии управления отходами на предприятии на основе логического вывода на онтологической модели с использованием языка семантических запросов.
\item Разработана архитектура и реализована интеллектуальная система поддержки принятия решений для генерации эффективной стратегии управления отходами на предприятии на основе описанных моделей и алгоритма. Проведено тестирование системы и проверка соответствия полученой стратегии управления отходами с данными по учреждению ГБУЗ «Николаевское ЦРБ». Стратегия управления отходами, полученная в результате генерации системой, соответствует сформулированным критериям эффективности и соответствует выбранным методом управления отходами, примененным учреждением ГБУЗ «Николаевское ЦРБ». Данный результат позволяет сделать вывод об эффективности разработанных моделей и алгоритма.
\end{itemize}
}

\newcommand{\CommonKeywords}{Ключевые слова: управление отходами, поддержка принятия решений, метаонтология, OWL-DL, логический вывод на онтологии, RDF, SPARQL, интеллектуальная система поддержки принятия решений}
\newcommand{\CommonKeywordsEng}{Keywords: waste management, decision support, Metaontology, OWL-DL, inference logical consequences, RDF, SPARQL, Intelligent Decision Support System}

\newcommand{\CommonPublicationsVAK}{
\begin{enumerate}
\item Кульцова,~М.Б. Интеллектуальная поддержка принятия решений по управлению отходами на городских территориях на основе онтологической модели представления знаний~/ Кульцова~М.Б., Руднев~Р.Ю., Жукова~И.Г., Аникин~А.В.~//~Изв. ВолгГТУ. Серия «Актуальные проблемы управления, вычислительной техники и информатики в технических системах». Вып. 13~:~межвуз. сб. науч. ст.~/~ВолгГТУ.~--~Волгоград,~2015.~--~№ 13 (117).~-- C.~104-109.
\end{enumerate}
}

\newcommand{\CommonPublicationsOther}{
\begin{enumerate}
\setcounter{enumi}{1}

\item Руднев,~Р.Ю. Онтологический подход к поддержке принятия решений по управлению отходами на городских территориях~/~Руднев~Р.Ю., Кульцова~М.Б., Жукова~И.Г.~//~XII Международная научно-практическая конференция «Инновации на основе информационных и коммуникационных технологий» ИНФО-2015 (Сочи, 1-10 окт. 2015 г.)~:~сб. науч. ст.~--~Сочи, 2015.~--~С.~568-571.

\item Руднев,~Р.Ю. Концепция поддержки принятия решений по управлению отходами на городских территориях на основе онтологического и имитационного моделирования~/~Руднев~Р.Ю., Кульцова~М.Б.~//~XX региональная конференция молодых исследователей Волгоградской области (Волгоград, 10-13 нояб. 2015 г.)~:~тез. докл.~/~отв. ред. А.В. Навроцкий~;~Волгогр. гос. техн. ун-т [и др.].~--~Волгоград, 2016.~--~C.~143-144.

\item Kultsova~M. An ontology-based approach to intelligent support of decision making in waste management~/~Kultsova~M., Rudnev~R., Anikin~A., Zhukova~I.~//~CIT DS Creativity in Intelligent Technologies Data Science, The 7th International Conference on Information, Intelligence, Systems and Applications,~IISA2016,~Greece,~2016 (принята к публикации).

\end{enumerate}
}

\newcommand{\CommonPublications}{
\begin{enumerate}

\item [1] Кульцова,~М.Б. Интеллектуальная поддержка принятия решений по управлению отходами на городских территориях на основе онтологической модели представления знаний~/ Кульцова~М.Б., Руднев~Р.Ю., Жукова~И.Г., Аникин~А.В.~//~Изв. ВолгГТУ. Серия «Актуальные проблемы управления, вычислительной техники и информатики в технических системах». Вып. 13~:~межвуз. сб. науч. ст.~/~ВолгГТУ.~--~Волгоград,~2015.~--~№ 13 (117).~-- C.~104-109.

\item [2] Руднев,~Р.Ю. Онтологический подход к поддержке принятия решений по управлению отходами на городских территориях~/~Руднев~Р.Ю., Кульцова~М.Б., Жукова~И.Г.~//~XII Международная научно-практическая конференция «Инновации на основе информационных и коммуникационных технологий» ИНФО-2015 (Сочи, 1-10 окт. 2015 г.)~:~сб. науч. ст.~--~Сочи, 2015.~--~С.~568-571.

\item [3] Руднев,~Р.Ю. Концепция поддержки принятия решений по управлению отходами на городских территориях на основе онтологического и имитационного моделирования~/~Руднев~Р.Ю., Кульцова~М.Б.~//~XX региональная конференция молодых исследователей Волгоградской области (Волгоград, 10-13 нояб. 2015 г.)~:~тез. докл.~/~отв. ред. А.В. Навроцкий~;~Волгогр. гос. техн. ун-т [и др.].~--~Волгоград, 2016.~--~C.~143-144.

\item [4] Kultsova~M. An ontology-based approach to intelligent support of decision making in waste management~/~Kultsova~M., Rudnev~R., Anikin~A., Zhukova~I.~//~CIT DS Creativity in Intelligent Technologies Data Science, The 7th International Conference on Information, Intelligence, Systems and Applications,~IISA2016,~Greece,~2016.

\end{enumerate}
}