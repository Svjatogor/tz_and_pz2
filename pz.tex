\documentclass[a4paper,english,russian]{G2-105}
\usepackage[T1]{fontenc}
\usepackage{listings}
\usepackage{graphicx}
\usepackage{longtable}
\usepackage{booktabs}
\usepackage{standalone}
\usepackage{newclude}
\usepackage[final]{pdfpages}
\usepackage{multirow}
\usepackage{sansmath} % Enables turning on sans-serif math mode, and using other environments
\sansmath % Enable sans-serif math for rest of document
\usepackage[math]{blindtext}

%\usepackage[utf8]{luainputenc}

\begin{document}
%\VSTUAddChapterWordToTOC % обязательно для ПЗ в магистерских диссертациях
\abstract{Аннотация}
\par Документ представляет собой дипломную работу к выпускной работе бакалавра на тему «Продвижение информационных товаров и услуг с применением интернет-маркетинга на примере корпорации Intel», выполненную студентом группы ММОП.м-3.1.3.2, Мельниковым Тимофеем Алексеевичем.
\par В данной работе рассмотрены методы и инструменты интернет-маркетинга, его использование в копропации Intel, а так же эффективность внедрения инструмента интернет-маркетинга Google Analytics в копрорацию Intel.
\par Объём дипломной работы составил составил \totalpages~страниц и включает \totalfigures~рисунков и \totaltables~таблицы. 

\newpage
\par Продвижение информационных товаров и услуг с применением интернет-маркетинга на примере корпорации Intel
%\newpage
\tableofcontents
\newpage

\starchapter{Введение}
\par В настоящее время во всех сферах человеческой жизни происходят значительные изменения. Не является исключением и маркетинг. Сейчас традиционный маркетинг меняется, появляются и создаются новые более востребованные маркетинговые инструменты. Результатом таких изменений можно считать появление интернет-маркетинга. Подавляющая часть потребителей становится активными пользователями сети интернет, что заставило компании переориентировать свою деятельность в интернет-сферу. Смена ориентации деятельности позволяет современным компаниям не только четко выбирать целевую аудиторию, но и эффективно взаимодействовать с нею, минимизируя при этом затраты.
\par Тема использования интернет-маркетинга для продвижения товаров и услуг является актуальной, так как в современном информационном обществе интернет-маркетинг стал наиболее эффективным инструментом привлечения потребителей, продвижения товара и является огромной базой для проведения различных исследований.
\par Объект --- использование сети интернет в маркетинге и применение интернет-маркетинга в корпорации Intel.
\par Предмет --- эффективные технологии и инструменты интернет-маркетинга.
\par Цель дипломной работы --- изучение основ интернет-маркетинга, его методов и инструментов, применение методов в корпорации Intel и анализ рентабельности применения инструмента Google Analytics в компании Intel.
\par В соответствии с поставленной целью мною определены следующие задачи:
\begin{itemize}
	\item Изучить особенности интернет-маркетинга, определить его понятие, цель и функции.
	\item Рассмотреть инструменты интернет-маркетинга, а также его преимущества и недостатки.
	\item Проанализировать использование методов интернет-маркетинга в корпорации Intel.
	\item Показать эффективность внедрения программы Google Analytics на примере корпорации Intel.
\end{itemize}
\par Данная работа состоит из введения, трех глав --- теоретической (Общая характеристика Интернет-маркетинга), аналитической (Использование методов интернет-маркетинга в корпорации Intel) и практической (Применение системы Google Analytics в корпорации Intel).

\newpage

\chapter{Общая характеристика Интернет-маркетинга}
\section{Особенности маркетинга в Интернете}
\par Интернет-маркетинг (англ. internet marketing) - это практика использования всех аспектов традиционного маркетинга в Интернете, затрагивающая основные элементы: цена, продукт, место продаж и продвижение. Основная цель - получение максимального эффекта от потенциальной аудитории.
\par Основные элементы комплекса интернет-маркетинга:
\begin{itemize}
	\item Товар (Product) --- то, что продается с помощью Интернета, должно иметь достойное качество. Он 			конкурирует не только с другими сайтами, но и традиционными магазинами.
	
	\item Цена (Price) --- принято считать, что цена в Интернете ниже, чем в обычном магазине за счет 				экономии на издержках. Контролируйте цены и сравнивайте их с конкурентами регулярно.
	
	\item Продвижение (Promotion) --- комплекс мер по продвижению как сайта, так и товара в целом в сети. 			Включает в себя огромный арсенал инструментов (поисковое продвижение, контекстная реклама, баннерная 			реклама, e-mail маркетинг, вирусный маркетинг, скрытый маркетинг, интерактивная реклама, работа с 				блогами и т.д.).
	
	\item Место продаж (Place) --- точка продаж, то есть сайт. Огромную роль играет как графический дизайн, 			так и юзабилити сайта, и качество обработки заявок с сайта. Так же стоит обратить внимание на скорость 		загрузки, работу с платежными системами, условия доставки, работу с клиентами до, во время и после продажи.
\end{itemize}
\par Интернет-маркетинг является составляющей электронной коммерции. Его также называют online-маркетингом. Он может включать такие части, как интернет-интеграция, информационный менеджмент, PR, служба работы с покупателями и продажи. Электронная коммерция и интернет-маркетинг стали популярными с расширением доступа к интернету и являют собой неотъемлемую часть любой нормальной маркетинговой кампании. Сегмент интернет-маркетинга и рекламы растёт как в потребительском секторе, о чем свидетельствует появление с каждым днем все новых интернет-магазинов. Основными преимуществами интернет-маркетинга считаются интерактивность, который ведет к максимальному повышению таких показателей как конверсия сайта и ROI интернет -рекламы. Интернет-маркетинг включает в себя такие элементы системы как:
\begin{itemize}
\item Gоисковый маркетинг в целом и SEO в частности
\item Продвижение в социальных сетях: SMO и SMM
\item Прямой маркетинг с использованием email, RSS и т.п. [24]
\end{itemize}
\par Основным слагаемым успеха является план маркетинга. Разработанный и внедренный владельцем в любом современном коммерческом предприятии, будь то традиционный магазин или электронный. Потребуются дополнительные маркетинговые мероприятия как в Сети, так и за ее пределами.
\par Забыв о некоторых особенностях пользователей интернета, служащих порой причиной дополнительных ограничений, их культуре и привычной манере общения, можно допустить вторую ошибку, рекламируя свой магазин с помощью рассылки по электронной почте всем, кто только встретится on-line, бесчисленных сообщений о его открытии. Это приведет к широкомасштабному и немедленному "наказанию" со стороны тех пользователей, которые терпеть не могут коммерцию в Сети. Есть, однако, и корректные способы рекламы своего бизнеса в интернете.
\par Надо уметь представить на рынке товары и услуги; необходимо также решить все связанные с этим задачи: сегментирование рынка, определение потребностей потребителей в целевых сегментах и способа продвижения товара, связь с потребителями (другими словами, реклама).
\par Понятие маркетинга в интернете остается наименее изученным и представляет главную проблему фирмы, решившей заниматься коммерцией в этой области. И хотя вряд ли кто-нибудь в ближайшем будущем сможет дать четкое определение данного термина (так как среда пользователей и технология еще не окончательно сформировались), уже сейчас можно предложить несколько стратегий ведения бизнеса в Сети.
\par Интернет создавался не с коммерческой целью, а для обмена информацией между учеными. Но идея ведения бизнеса не чужда ей - она была заложена в самой структуре Сети, хотя привычные для нас красивые названия, такие как "торговые центры", "стендовая реклама", "стратегическое положение", практически ничего не значат в мире электронном. Нехватка новых терминов и обозначений сегодня уже не является временным неудобством в определении маркетинговых подходов, а ставит перед нами вопрос "Что такое маркетинг?", отвечать на который нужно совершенно по-новому.
\par Проблемы маркетинга в интернете. Под маркетингом обычно подразумевается изучение рынка (размеров, демографических характеристик, потребностей) для размещения продукта, определения цены, вероятных покупателей и выработки способов общения с последними. Поэтому человек, занимающийся маркетингом в интернете, обычно сталкивается со следующими проблемами:
\begin{itemize}
\item Неизвестными размерами рынка. О пользователях интернета знаем очень мало. Невозможно, даже более или менее точно определить их число. Наиболее распространенная оценка размера интернета, принадлежащая Internet Society, определяется как число подключенных к сети узловых компьютеров. Вычислить количество пользователей в зависимости от серверов так, чтобы результат соответствовал действительности, практически невозможно. По некоторым оценкам, в настоящее время пользователей Сети порядка 50 миллионов.
\item Пассивностью потребителей. Достаточно ли знать, что потребителей "много" и "их число растет". Как правило (имеется в виду разновидности бизнеса), точные цифры --- "сколько" и "как быстро" - не нужны. В конце концов, расходы на подключение к интернету по сравнению с затратами на открытие настоящего магазина и оплату труда работников относительно невелики.
\par Настоящие проблемы возникают, когда вы пытаетесь сообщить (неизвестно кому!) о своем существовании и продукции. Сегодняшние возможности передачи данных - электронная почта и доски объявлений (телеконференции) --- абсолютно неприемлемы для распространения такой информации. Необходимо четко понимать, что товары и услуги нельзя рекламировать в Сети так же, как по телевидению, то есть прямо и настойчиво. Предложения своих услуг, продукции в подобной форме и явное продвижение самого себя не поощряются. Нарушение неписаных правил немедленно приводит к реакции со стороны пользователей --- нарушителя "сжигают", иными словами, ему посылают тысячи осуждающих сообщений по электронной почте. Огромный объем информации способен вывести из строя сеть пользователя, которая окажется не в состоянии справиться с обработкой такого количества сообщений. Это отобьет у нарушителя всякое желание иметь дело с интернетом. Пользователи могут также объявить бойкот товарам и услугам компании и вообще перестать связываться с ней по Сети. И уж совсем редко, но все же случается и такое, что нарушителю лично доставляют массу беспокойства телефонными звонками часа в два ночи домой, звонками на работу.
\par Например, при создании крупнейшей американской коммерческой сети учитывалась возможность подобных операций. Однако, пройдя и через рекламу, и через "торговые центры", компания изменила структуру получения доходов, когда выяснилось, что пользователям нужно, скорее, средство общения, нежели возможность совершать покупки в режиме online. А в некоторых случаях --- таких как торговля цветами или программным обеспечением --- дела шли как нельзя лучше. Так что успех предприятия в интернете зависит не столько от умения торговца правильно подать себя, сколько от того, окажутся ли полезными его товары или услуги для пользователей.
\par Существует множество объяснений, почему некоторые электронные магазины терпят неудачу (здесь не учитывается явные недостатки: плохой пользовательский интерфейс, нехватки графики, неудобного механизма оформления заказов, невозможности оплаты наличными и так далее). Однако, как правило, поведение покупателей обуславливается следующими моментами:
\begin{enumerate}
\item привычки: "Обычно я делаю покупки (хлеб, одежду и так далее) совсем не так";
\item несоответствие цели: "Я сюда не за этим пришел";
\item неизвестность: "Я не знаю, что там было, --- я искал только то, что хотел найти";
\item несовершенство систем поиска: "Мне был нужен видеомагнитофон, но я не собирался просматривать десять разных магазинов".
\end{enumerate}
\par По сравнению с обычными средствами массовой информации - газетами, телевидением и радио, - которые, по сути, предназначены для передачи коммерческих сообщений, интернет довольно пассивен в смысле проведения маркетингового комплекса. Вы просто делаете вывеску и ждете посетителей.
\item Незнание потребителей. Если разобрались с незнанием реальных объемов рынка сбыта и потребительской пассивностью, то что сказать о маркетинге, основанном на достигнутых результатах и обратной связи с покупателями. Другими словами, не достаточно ли определить один раз маркетинговые мероприятия и проводить их, основываясь на собственном опыте, пусть и небольшом. Можно не изменять, не адаптировать политику маркетинга, рекламу, основанную лишь на отзывах посетителей.
\par В отличие от пассивной, нисходящей на потребителя модели маркетинга, интернет позволяет осуществить взаимодействие поставщиков и потребителей, при котором последние сами становятся поставщиками (в частности, поставщиками информации о своих потребностях).
\end{itemize}
\par Поскольку интернет представляет собой совершенно новую коммуникационную среду в отличие от традиционных средств информации, во многих случаях некоторые приемы и средства маркетинга не могут быть применены в их существующей форме.
\par При рассмотрении модели, использующей для рекламы традиционные средства массовой информации, оказывается, что использование интернета дает возможность потенциальным клиентам не выступать в роли пассивной аудитории, а самостоятельно принимать решение, следует ли им знакомиться с конкретной рекламной информацией. Таким образом, интернет трансформирует функцию маркетинга.
\par При работе в интернете фирма, раскрывая и удовлетворяя потребности клиента, должна стремиться внести свой вклад в разработку новых идей и методов для электронной коммерции.
\par Таким образом, новая роль маркетинга помимо удовлетворения потребностей клиента непосредственно включает в себя "альтруистическую", кооперативную цель облегчения развития рынка.
\section{Основная цель интернет-маркетинга}
\par Интернет-маркетинг появился в начале 1990-х годов, когда текстовые сайты начали размещать информацию о товарах. Сейчас интернет-маркетинг --- это нечто большее, чем продажа информационных продуктов, сейчас идет торговля информационным пространством, программными продуктами, бизнес-моделями и многими другими товарами и услугами. Такие компании, как Google, Yahoo, и MSN подняли на новый уровень и сегментировали рынок интернет-рекламы, предлагая малому и среднему бизнесу услуги по локальной рекламе. Рентабельность инвестиций возросла, а расходы удалось снизить. Этот тип маркетинга стал основой современного капитализма, которая позволяет любому, у кого есть идея, товар или услуга, достичь максимально широкой аудитории. Использование термина "интернет-маркетинг" обычно подразумевает использование стратегий маркетинга прямого отклика, которые традиционно используются при прямых почтовых рассылках, радио и в телевизионных рекламных роликах, только здесь они применяются к бизнес пространству интернета. Эти методы оказались очень эффективными при использовании в интернете благодаря возможностям точно отслеживать статистику, умноженным на возможность находиться в относительно постоянном контакте с потребителями. Эта возможность анализа применяется сейчас повсеместно, и поэтому так часто можно увидеть такие термины, как ROI --- коэффициент окупаемости инвестиций, conversion rate --- коэффициент эффективного посещения (он же - конверсия сайта), а также мгновенно получить статистику продаж, спроса и т.д. [24]
\par Основная цель интернет - маркетинга - это получение максимального эффекта от потенциальной аудитории. Условно, все инструменты интернет-маркетинга можно разделить на две группы:
\begin{itemize}
\item Поисковый маркетинг, куда относятся поисковое продвижение сайта, контекстная реклама.
\item Не поисковый маркетинг, куда можно отнести медийную рекламу, вирусный маркетинг, маркетинг в социальных сетях, PR в интернете.
\end{itemize}
\par Отдельно стоит упомянуть веб-аналитику, которая показывает, насколько эффективно работает маркетинг, окупаются ли средства вложенные в продвижение и многое другое. Интернет-маркетинг, как комплекс мер по работе с сайтом выходит за рамки собственно привлечения трафика на сайт. Это не просто продвижение сайта, это уже продвижение бизнеса в сети интернет. Весь комплекс работ в рамках интернет-маркетинга имеет три основные цели:
\begin{itemize}
\item информирование аудитории о услугах/продукте, его преимуществах;
\item вызвать интерес к услуге/продукту;
\item "подтолкнуть" заинтересовавшихся к покупке.
\end{itemize}
\par Добиться реализации всех трех целей, путем отдельно взятой рекламной компанией или тематическим продвижением --- чрезвычайно сложно, а вот сочетая все доступные инструменты и возможности --- реально. Всю работу по продвижению бизнеса в интернете можно условно разделить на следующие этапы:
\begin{enumerate}
\item Знакомство с бизнесом. На этом этапе проводится различного рода исследования. Необходимо понять, что и кому будем предлагать, определить характеристики продукта и компании, целевую аудиторию и конкурентное окружение. На этом же этапе идет изучение истории проекта: что делалось ранее и что делается сейчас, какие цели ставились и какие результаты получены.
\item Подготовка к продвижению. На этом этапе планируется компания по продвижению: определение методов и способов продвижения, аудит сайта, расчет бюджета, определение всех необходимых работ по подготовке сайта, определение основных показателей эффективности работы сайта, рекламных компаний. Внедрение и настройка инструментов веб-аналитики.
\item Подготовка к продвижению. На этом этапе планируется компания по продвижению: определение методов и способов продвижения, аудит сайта, расчет бюджета, определение всех необходимых работ по подготовке сайта, определение основных показателей эффективности работы сайта, рекламных компаний. Внедрение и настройка инструментов веб-аналитики.
\item Формирование отчетности, анализ эффективности, внесение изменений в план действий. С самого начала ведется контроль эффективности предпринимаемых мер и действий, что позволяет во время сфокусировать внимание на проблемных участках и в конечном итоге - добиваться всех поставленных целей.
\end{enumerate}
\section{Интернет-маркетинг и традиционный маркетинг: сходства и различия}
\par Использование Интернета привносит новые особенности и преимущества по сравнению с маркетингом, основанном на традиционных технологиях. Вот некоторые из них:
\begin{itemize}
	\item Переход ключевой роли от производителей к потребителям.
\par Одним из наиболее фундаментальных качеств, привнесенных интернетом в мир современной коммерции, является переход ключевой роли от производителей к потребителям. Интернет сделал реальностью для компаний возможность привлечь внимание нового клиента всего за десятки секунд, проведенных им перед экраном компьютера. Однако в то же время он дал возможность тому же пользователю за несколько щелчков мыши перейти к любому из конкурентов. В такой ситуации внимание покупателей становится самой большой ценностью, а установленные взаимоотношения с клиентами главным капиталом компаний.
	\item Глобализация деятельности и снижение транзакционных издержек.
\par Интернет значительно изменяет пространственный и временной масштабы ведения коммерции. Он является глобальным средством коммуникации, не имеющим каких-либо территориальных ограничений, при этом стоимость доступа к информации не зависит от удаленности от нее, в противоположность традиционным средствам, где эта зависимость прямо пропорциональна. Таким образом, электронная коммерция позволяет даже самым мелким поставщикам достигать глобального присутствия и заниматься бизнесом в мировом масштабе. Соответственно, заказчики также получают возможность глобального выбора из всех потенциальных поставщиков, предлагающих требуемые товары или услуги независимо от географического расположения. Расстояние между продавцом и покупателем играет роль лишь с точки зрения транспортных издержек уже на этапе доставки товаров.
\par Временной масштаб в интернете также значительно отличается от обычного. Высокая эффективность коммуникативных свойств интернета обеспечивает возможность сокращения времени на поиск партнеров, принятие решений, осуществление сделок, разработку новой продукции, и т. д. Информация и услуги в интернете доступны круглосуточно. Кроме того, его коммуникативные характеристики обладает высокой гибкостью, позволяющей легко производить изменения представленной информации, и, тем самым, поддерживать ее актуальность без временной задержки и затрат на распространение.
\par Названные эффекты также приводят к значительному сокращению транзакционных издержек, то есть издержек, связанных с налаживанием и поддержанием взаимодействия между компанией, ее заказчиками и поставщиками. При этом стоимость коммуникаций, по сравнению с традиционными средствами, становится минимальной, а их функциональность и масштабируемость значительно возрастают.
\item Персонализация взаимодействия и переход к маркетингу "один-одному".
\par Используя средства электронного взаимодействия, компании могут получать подробную информацию о запросах каждого индивидуального заказчика и автоматически предоставлять продукты и услуги, соответствующие индивидуальным требованиям. Одним из простых примеров может служить персональное представление web-сайта для каждого из клиентов или партнеров компании.
\par В результате интернет позволяет перейти от массового маркетинга к маркетингу "один-одному". В таблице ~\ref{am} приведены данные по сравнению характеристик массового маркетинга с маркетингом "один-одному".
\begin{longtable}{|c|c|}
    \caption{Сравнение массового маркетинга и маркетинга "один-одному"}\\ \hline
    \label{am} 
    Массовый маркетинг & Маркетинг "один к одному" \\ \hline \endhead
    Усредненный покупатель & Отдельный покупатель \\ \hline
    Анонимность покупателя & Характеристики покупателя \\ \hline
    Стандартный продукт & Специальное маркетинговое предложение \\ \hline
    Массовое производство & Специальное производство \\ \hline
    Массовое распределение & Индивидуальное распределение \\ \hline
    Массовая реклама & Индивидуальное обращение \\ \hline
    Массовое продвижение & Индивидуальные стимулы \\ \hline
    Одностороннее обращение & Двусторонние обращения \\ \hline
    Масштабная экономика & Целевая экономика \\ \hline
    Доля рынка & Доля покупателей \\ \hline
    Все покупатели & Потенциально прибыльные покупатели \\ \hline
    Привлечение покупателей & Удержание покупателей \\
\end{longtable}
\item Снижение трансформационных издержек
\par Снижение трансформационных издержек может достигаться за счет оптимального выбора структуры товарного ассортимента, сокращения времени на разработку и внедрение новой продукции, обоснованной политики ценообразования, снижения числа посредников, затрат на сбыт и т.д.
\par Например, одним из способов снижения трансформационных издержек может быть сокращение каналов распространения товаров. Причиной сокращения каналов распространения является возможность для фирм взять на себя функции, традиционно выполняемые специалистами промежуточных звеньев, так как интернет обладает более эффективной возможностью взаимодействия с потребителями и одновременно позволяет отслеживать информацию о потребителях.
\par Особый случай --- продукты и услуги, которые могут быть доставлены электронным способом. При этом путь доставки сокращается максимально. Электронный способ широко применяется для доставки цифровых продуктов индустрии развлечений (фильмы, видео, музыка, журналы и газеты), информации, средств обучения и эффективно используется компаниями, занимающимися разработкой и поставкой программного обеспечения.
\end{itemize}
\section{Преимущества и ограничения интернет - маркетинга}
\par Интернет-маркетинг в первую очередь предоставляет потребителю возможность получить информацию о товарах. Любой потенциальный потребитель может, используя интернет, получить информацию о товаре, а также купить его. Хотя, если там не будет информации об одном товаре, или он её не найдёт, то, скорее всего он приобретёт другой товар у конкурента.
\par Применение методов интернет-маркетинга нацелено на экономию средств (на заработной плате сотрудников отделов продаж и на рекламе), а также на расширение деятельности компаний (переход с локального рынка на национальный и международный рынок). При этом как крупные компании, так и малые, имеют более уравновешенные шансы в борьбе за рынок. В отличие от традиционных рекламных медиа (печатных, радио и телевидения), вход на рынок через интернет является не слишком затратным. Важным моментом является то, что в отличие от традиционных маркетинговых методов продвижения, интернет-маркетинг дает чёткую статистическую картину эффективности маркетинговой кампании.
\par В сравнении с другими видами медиамаркетинга (печатными, радио и телевидением), интернет-маркетинг растет очень быстро. Он завоёвывает все большую популярность не только у бизнеса, но и обычных пользователей, которые хотят продвинуть свой эффективный веб-сайт или блог и заработать на нем. Тем не менее, в развитых странах, затраты на интернет-маркетинг и рекламу составляют около 5\% от общих рекламных затрат.
\par Следующее неудобство состоит в том, что интернет-маркетинг не дает возможность потребителю опробовать товар до того, как сделать покупку. Но большинство потребителей решают эту проблему просто. Они знакомятся с интересующим их товаром в обычном магазине, а покупку делают в интернет-магазине. Германия, например, приняла в 2000 году закон (Fernabsatzgesetz, позже объединён с BGB), по которому любой покупатель может вернуть товар, купленный через интернет без всяких объяснений и получить полный возврат денег. Это одна из основных причин, почему в Германии так развита интернет-торговля.
\par Проблема отсутствия возможности у покупателя "потрогать" товар также может решаться иными способами, например, некоторые владельцы интернет магазинов используют фотографии товара высокого качества и разрешения, стараясь передать в изображениях все детали и особенности своей продукции. Набирает популярность и использование специальной фото-техники для оцифровки снимков товара в формате 3D (объемное изображение), дающее посетителю интернет магазина рассмотреть товар со всех ракурсов
\section{Влияние интернет маркетинга на бизнес}
\par Интернет-маркетинг оказал огромное влияние на ряд деловых сфер, включая музыкальную индустрию, банковское дело, рынок портативных электронных устройств (мобильные телефоны, плееры и т.д.), так называемый "блошиный рынок" и главное --- на рекламу.
\par В музыкальной индустрии на данный момент жесткие носители потеряли свое доминирующее место. Музыкальные произведения доступны онлайн, на удаленных серверах.
\par Интернет-маркетинг также повлиял и на банковскую индустрию. Онлайн-банкинг является более удобным для клиента, так как избавляет от необходимости посещать каждый раз банк или его филиалы. В США на сегодняшний день большинство людей пользуются услугами онлайн-банкинга. Онлайн-банкинг является одним из наиболее быстрорастущих секторов интернет-бизнеса. Увеличивающиеся скорости интернет-соединений занимают в этом исключительно важную роль. Из всех пользователей Интернета около 70\% пользуются услугами интернет-банкинга.
\par Интернет-аукционы завоевали популярность, блошиные рынки борются за выживание. Уникальные вещи, которые раньше можно было найти на блошиных рынках, теперь продаются на онлайн-аукционах, таких как eBay. Также развитие аукционов сильно повлияло на цены на уникальные и антикварные вещи. Если прежде информацию о цене найти было трудно, то теперь можно посмотреть цену на аналогичную вещь на аукционе. И иметь хотя бы общее представление о стоимости товара, так как всегда можно узнать, за сколько продавалась та или иная вещь. Все больше и больше продавцов подобных товаров ведут свой бизнес онлайн, сидя дома.
\par Эффект на рекламную индустрию был и остается поистине огромным. В течение всего нескольких лет объем онлайн-рекламы стремительно вырос и достиг десятков миллиардов долларов в год. Рекламодатели начали активно менять свои предпочтения и сегодня интернетреклама уже занимает большую рыночную нишу, чем реклама на радио (в развитых странах).
\par На сегодняшний день сложно найти крупное индустриальное предприятие, которое не продвигает себя в сети. Тенденции роста можно легко увидеть и по постоянному расширению торговых интернет-площадок, а также росту их количества. Торговые онлайн-площадки уже давно перестали быть досками объявлений, из которых они и выросли. Сегодня некоторые из них превратились в крупные корпорации, предоставляющие целый ряд маркетинговых услуг. Растут и цены за участие на таких площадках (имеется в виду привилегированное членство), несмотря на то, что количество их увеличивается.
\section{Проблемы интернет-маркетинга в России}
\par Главными объективными препятствиями на пути становления маркетинга в России стали супермонополизм в промышленности, диктат централизованного ценообразования, дефицит товаров и неготовность кадров к работе в многовариантном, многополюсном, взаимозависимом мире. Все это консервировало несвободу выбора как для покупателей, так и во многом для производителей товаров и услуг. В таких условиях потенциал маркетинга не мог быть реализован, за исключением отдельных шагов па уровне отдельных фирм, организаций.
\par Субъективными факторами, тормозившими развитие маркетинга, стали распространенные в нашем обществе антимаркетинговые стереотипы (психологические установки и подходы) в восприятии маркетинга сo стороны хозяйствующих субъектов и граждан, либо неоправданно упрощавшие его понимание и процедуры осуществления, либо наоборот, излишне усложнявшие и приводившие к отказу от его использования целые сферы экономики - малый бизнес, некоммерческие виды деятельнoсти и др. Эти стереотипы в значительной степени не преодолены и до сих пор.
\section{Web-сайт --- как инструмент ведения бизнесса}
\par Организация представительства в Интернете – общепринятый инструмент ведения бизнеса. Профессионально разработанный web-сайт может служить как высокоэффективным инструментом ведения бизнеса, так и информационным или имиджевым ресурсом, рассказывающим о деятельности фирмы.
\par Задачи, которые решают сайты, многообразны – от организации внутренних или внешних коммуникаций компании (корпоративные ресурсы) и интернет-торговли (интернет-магазины) до самовыражения (сетевые дневники) и объединения людей по интересам (комьюнитиобразующие ресурсы).
\par Ниже описаны наиболее распространенные виды. Отметим, что задача сайта – не только привлечь пользователей, но и заинтересовать их представленной информацией. Еще на этапе проектирования ресурса следует разработать стратегию, как удержать посетителей.
\par Пол Темпорал рекомендует придерживаться следующих правил1 при создании сайта:
\begin{enumerate}
\item быстрая загрузка;
\item качественное содержание;
\item легкость пользования и навигации;
\item простой, соответствующий аудитории язык;
\item немедленная обратная связь.
\end{enumerate}
\par Если ресурс понравится посетителям, они будут часто им пользоваться.

\subsection{Сайты-«визитки»}
\par Сайты-«визитки» --- это сайты начального уровня, состоящие, как правило, из одной-пяти страниц. Фактически это небольшое представительство компании в Интернете со сведениями о ней, кратким перечнем продуктов и услуг, контактной информацией. На сайтах-«визитках» могут быть размещены простейшие формы взаимодействия с посетителями (например, форма для отправки сообщений или гостевая книга). Обычно партнеры компании заходят на них, чтобы уточнить почтовый адрес или посмотреть схему проезда. В большинстве случаев сайты-«визитки» не считаются эффективными маркетинговыми каналами.
\par Создание таких сайтов относительно дешево (от 5 тыс. рублей), однако сайты такого уровня практически не подлежат развитию. В силу простоты сайтов-«визиток» при их создании можно обходиться без установки системы управления контентом. На разработку подобного сайта у специалистов обычно уходит всего лишь несколько дней.
\par Существуют примеры, когда сайты-«визитки» имели большое имиджевое значение. Например, разработанный студией Артемия Лебедева сайт адвоката Павла Астахова (www.astakhov.ru) содержит всего одну страницу (рис. 1). Кроме адреса электронной почты, на нем нет другой информации. За счет оригинального минимализма в сочетании с талантливой креативной идеей сайт адвоката выполняет функцию промо-ресурса. Соответственно, и стоимость его сопоставима с ценой разработки промо-ресурсов.

\subsection{Промо-ресурсы}
\par Промо-ресурсами называют сайты, рекламирующие отдельный продукт или услугу (узкий спектр продуктов или услуг) либо продвигающие бренд. Создание такого ресурса включает в себя создание стильного дизайна, индивидуальных систем навигации, подключение необходимых программных модулей.
\par Такие сайты напоминают рекламные ролики. Действительно, основные требования, которым должны соответствовать промо-ресурсы, – яркость, красочность и запоминаемость. Это, в частности, достигается за счет flash-анимации. Посетители часто заходят на такие сайты повторно, потому что они «красивые».
\par Эффективное использование промо-ресурса возможно только в совокупности с мощной рекламной кампанией (как правило, баннерной или рич-медиа либо с использованием нестандартных рекламных носителей). Благодаря ему можно быстро раскрутить рекламируемую услугу (товар, бренд), а также использовать его в качестве источника справочной информации.
\par На разработку подобного ресурса требуется несколько месяцев, а затраты в ряде случаев могут составить сотни тысяч рублей.

\subsection{Корпоративные ресурсы}
\par Корпоративные интернет-ресурсы – эффективные инструменты для привлечения и обслуживания клиентов. Они предоставляют посетителям максимально полную информацию по реализуемым продуктам (услугам) по возможности с описаниями как самих продуктов (услуг), так и схем взаимодействия с потенциальными клиентами (рис. 5). Обычно в корпоративных ресурсах имеются средства интерактивного общения посетителей и персонала компании.
\par Эффективное использование корпоративных ресурсов предполагает комбинирование недорогих рекламных мероприятий (оптимизацию под поисковые системы) с разовыми рекламными акциями. При правильном подходе в течение года корпоративный ресурс может стать основным средством привлечения новых и удержания существующих клиентов.
\par В книге «Эффективный бренд-менеджмент» приводятся примеры, поясняющие, каких целей хочет добиться компания, размещая информацию о своих продуктах и услугах в Интернете.
\begin{itemize}
\item Построить виртуальный бизнес или поддержать свой традиционный.
\item Предоставить клиентам специфическую техническую информацию.
\item Увеличить сбыт, сократить торговый персонал или повысить его эффективность.
\item Предложить клиентам альтернативный канал покупки/распространения.
\item Разработать программу управления взаимоотношениями с клиентами.
\item Привлечь новый сегмент потребителей. Выйти на мировой рынок.
\item Повысить уровень осведомленности о марке и уровень ее признания.
\end{itemize}
\par Любая из этих целей (или иная) может быть стратегической. Пол Темпорал убежден, что важно сконцентрироваться на поставленной цели и, исходя из нее, определить, каким должен быть сайт. С учетом намеченных задач, в частности, разрабатывается дизайн сайта.
\subsection{Корпоративные медиа}
\par Особого разговора заслуживают корпоративные медиаресурсы. Они находятся между корпоративными и информационными ресурсами и часто используются для организаци эффективных внутренних коммуникаций в фирме. Во многих случаях это интернет-версии бумажных корпоративных изданий – газет или журналов, дополненных возможностями для интерактивного общения сотрудников и клиентов. Они могут быть предназначены либо для служащих компании, либо для широкой аудитории.
\par Пример успешной реализации такого проекта – внутрикорпоративная электронная интранет-газета «PROВИДЕЦ» группы компаний Pro -Vision  (рис. 6). Ее неоднократно отмечали на специализированных конкурсах. В пресс-релизе Pro -Vision  приводилась цитата их президента Владимира Виноградова: «Наш бизнес строится на взаимодействии между людьми, и от того, насколько эффективно выстроена коммуникация внутри коллектива, зависит успех нашей работы. Важную роль в решении этой задачи играет „PROВИДЕЦ“. Основное преимущество нашей корпоративной газеты – это интерактив. Для сотрудников Pro-Vision  корпоративное издание это не только своевременное информирование о событиях внутри компании или в отрасли, каждый видит результат своего непосредственного участия в формировании информационного пространства».
\par Мероприятия, посвященные развитию корпоративных медиа (как печатных, так и сетевых), регулярно проводит Ассоциация корпоративных медиа России. О ее работе можно узнать на сайте www.corpmedia.ru.
\subsection{Товарные каталоги и интернет-магазины}
\par Товарные каталоги представляют собой корпоративные сайты фирм, предлагающих большой ассортимент товаров.
\par При создании такого сайта нужно специальное программное обеспечение для поддержки максимально простого добавления/удаления товаров в рубрики и подрубрики, быстрого создания разветвленной структуры товарных категорий, удобного ввода описаний товаров различных типов. Кроме того, на сайте хорошо бы разместить фотографии, схемы, презентации, подробную документацию, указать цены продукции и услуг. Посетители заказывают товары с помощью электронной почты, которую потом обрабатывает менеджер компании. В результате использования сайта-каталога полностью или частично отпадет необходимость в значительных затратах на издание бумажного каталога; клиенты смогут быстро находить нужные товары (рис. 7).
\par Интернет-магазины предназначены для продажи товаров или услуг через Сеть. В отличие от товарного каталога такие сайты содержат функции заказа и оплаты товаров через Интернет, а также работы с корзиной заказов. У интернет-магазинов есть система поиска по одной или многим характеристикам товара.
\par Для оценки эффективности работы интернет-магазина применяют коэффициент конверсии. Конверсия – отношение числа посетителей, воспользовавшихся предлагаемой на сайте услугой, к числу пользователей, пришедших на сайт. Согласно исследованию НАУЭТ и CNews Analytics  (2005), средний коэффициент конверсии в российском интернет-магазине составляет 4,9 %. 
\par Для увеличения коммерческой отдачи бывает полезно вести на сайте отдельную новостную ленту, посвященную работе данного сервиса. Желательно дать детальное описание товаров и предоставить посетителям возможность обсуждать на сайте достоинства и недостатки искомых моделей, сравнивать их (рис. 8).
\par Возможности интернет-торговли в сфере потребительских товаров ограничены. Например, женщина, покупающая модную одежду или обувь, наверняка захочет обойти весь магазин и примерить понравившиеся вещи. Альтернативу этому процессу в Интернете еще не придумали.
\par Говоря об организации интернет-торговли, следует упомянуть распространенные в Интернете на Западе аффилированные программы (см., например, Дэвид Филлипс (David Phillips). PR в Интернете. – М.: ФАИР-ПРЕСС). Он так описывает подобные программы: «Если вы занимаетесь торговлей в Интернете, то можете найти формы вознаграждения для других сайтов (аффилированных), которые направляют к вам поток пользователей. Если, появляясь на вашем сайте, они совершают ряд желательных вам действий (покупают предлагаемый товар), вы можете выплачивать владельцам аффилированных сайтов определенное вознаграждение. В качестве оплаты можно использовать и деньги, и информацию, включая интеллектуальную собственность».
\subsection{Информационные ресурсы}
\par Для ресурсов этого типа характерно большое количество информационных материалов (новости, аналитические статьи, репортажи, интервью, комментарии, обзоры и т. д.) по тематике (или тематикам) сайта. Они регулярно обновляются. Как правило, подобные сайты представляют собой сетевые издания, информационные агентства или интернет-представительства оф-лайновых СМИ (печатных изданий (рис. 9), новостных служб телевизионных каналов и радиостанций и т. д.).
\par Для информационных ресурсов обычно устанавливается система управления контентом, обеспечивающая возможность добавления/ редактирования информации большим числом пользователей. Она поддерживает разветвленную структуру сайта с добавлением новых рубрик, помогает максимально быстро и просто публиковать документы в любом количестве рубрик, акцентирует внимание на новых материалах (например, размещение промо-блоков на главной странице сайта).
\par Важные требования, которым должны соответствовать такие ресурсы, – быстрая загрузка страниц сайта и возможность работать с архивами публикаций. Зачастую именно информационные ресурсы атакуют хакеры, поэтому надо предъявлять повышенные требования к техническому обеспечению их безопасности.
\subsection{Сетевые дневники}
\par В сетевых дневниках, или блогах, представлены серии статей или заметок (постов), расположенных в обратном хронологическом порядке и написанных одним или несколькими авторами. Как правило, читателям предоставляется возможность оставлять комментарии. Коллективные сетевые дневники образуют так называемые сообщества.
\par Слово «блог» происходит от английского weblog («web-журнал»). Совокупность всех сетевых дневников называют блогосферой. Людей, ведущих блоги, называют блоггерами. В 2006 году в России насчитывалось более двух миллионов блогов, и количество их постоянно увеличивается. Однако русскоязычная блогосфера составляет всего лишь 3–4\%  мировой. Каждые несколько секунд в глобальном Интернете появляется новый блог.
\par Специальные службы для размещения дневников называются блог-хостингами. Любой посетитель, даже не имеющий высокой квалификации работы в Интернете, может с их помощью завести сетевой дневник. В Рунете используется около 30 крупных блог-хостингов. Американский Livejournal.com до сих пор очень популярен в наше стране, хотя ему уже наступает на пятки российский LiveInternet.ru.
\par Темы, обсуждаемые отечественными сообществами, весьма разнообразны. Наиболее популярны следующие: реклама, пиар, маркетинг, дизайн, кино, музыка, взаимоотношения полов и т. д. Многие российские журналисты ведут «живой журнал» и размещают в нем гиперссылки на свои публикации. Это в ряде случаев существенно увеличивает читательскую аудиторию издания. Часто журналисты используют собственные блоги как альтернативную основу для публикации, это помогает им лучше контролировать текст.1 Например, блоги дают им площадь для размещения статей, не соответствующих стандартному формату, или слишком субъективных, или посвященных вопросам, не упоминающимся в новостях или не представляющих интереса для редактора. В определенном смысле блогинг дает журналистам свободу, которая раньше была предоставлена лишь избранным обозревателям. Крайне важна интерактивность публикаций, к обсуждению текстов часто подключаются люди, упомянутые в статьях (рис. 10).
\subsection{Комьюнитиобразующие ресурсы}
\par В широком смысле комьюнити, или сообщество (англ. community),  – это постоянная аудитория любого интернет-сайта. Наличие устойчивого и многочисленного сообщества считается залогом успешного развитии web-проекта.
\par Роберт Блай в книге «Как стать гуру за 60 дней» пишет о причинах, по которым следует формировать сообщество на собственном web-сайте.1 Во-первых, это повышает привлекательность сайта, то есть определяет, сколько времени проводят люди на сайте, и частоту посещений. Во-вторых, позволяет обеспечить сайт функциональными возможностями для формирования сообщества можно и без значительных затрат. Иными словами, это способ повысить ценность сайта при весьма небольших инвестициях. И в третьих, как супервайзер или менеджер своего электронного сообщества вы можете наблюдать за всем, что происходит. Благодаря этому можно много узнать о ваших посетителях и о том, что они на самом деле хотят.
\par Для формирования электронных сообществ по интересам на собственном сайте можно использовать такие механизмы, как опросы, голосования, тесты и краткие анкеты, доски объявлений, чаты и гостевые книги и др.
\par Более корректно понимать под комьюнити группы людей, объединенных одними интересами и общающихся посредством Интернета. Возможность общения здесь играет ключевую роль, поэтому любые такие ресурсы должны содержать разнообразные интерактивные сервисы.
\par В Интернете легко найти людей со схожими взглядами, познакомиться в Сети проще, чем на улице. Именно этим обусловлено наличие в Интернете самых разнообразных сообществ (от владельцев кошек до любителей авиационных симуляторов).
\par Как отмечается в Википедии, с развитием сетевых коммуникаций интернет-сообщества начинают играть важнейшую роль в развитии бизнеса и экономики в целом. Это основа для формирования (или разрушения) брендов компаний, потому что в Сети появляются отзывы незаинтересованных людей, которым читатель доверяет. Успешные компании создают на своих сайтах он-лайн-клубы, а сообщества профессионалов становятся регуляторами рынка в различных областях. В них происходят обучение и обмен информацией, создают альянсы, находят новые решения для развития отрасли.
\par Комьюнитиобразующие ресурсы весьма разнообразны (рис. 11, 12). К примеру, к подобному типу сайтов можно отнести многопользовательские ролевые он-лайн-игры – аудитория наиболее успешных мировых проектов насчитывает несколько миллионов человек. Владельцы подобных ресурсов делают бизнес, в частности, на продаже специальных артефактов для этих игр. Виртуальные валюты часто можно переводить в реальные деньги.
\par Говоря об интернет-сообществах, часто используют термин Web 2.0 .2 С точки зрения Тима О'Рейли, известного издателя и активиста движения Web 2.0,  идея задействовать коллективный интеллект и «эффект Сети», чтобы мобилизовать неиспользуемые ресурсы, является фундаментальным аспектом изменений, которые происходят в Web 2.0,  но, главное, надо относиться к Интернету как к платформе. Его статья «What is Web 2.0», опубликованная в сентябре 2005 года, в значительной степени определила новый этап развития Интернета (переведенную статью можно прочитать в журнале «Компьютера» за тот же год).
\par До сих пор этот термин вызывает большие дискуссии в профессиональной среде. По мнению ряда специалистов, Web 2.0  – дутый феномен, ничего принципиально нового в использовании уже существовавших ранее технологий нет, и с чисто технической точки зрения они правы. Многие полагают, что Web 2.0  – спекуляция, которая закончится так же, как знаменитый «дот-ком бум», за которым последовал крах множества web-ориентированных компаний в 2001 году.
\par Участившееся использование термина специалистами в области PR приводит еще к большей путанице. Весьма показательны предварительные итоги голосования на сайте pr|online.  Большинство участников опроса (40 \%) заявили, что впервые слышат о подобном термине, 23 \%  считают его одним из средств массовых коммуникаций, а 16 \% – удачной выдумкой сетевых маркетологов (мы приводим здесь лишь три наиболее популярных ответа).
Хотя до сих пор большинство PR-специалистов не понимают, что означает Web 2.0,  некоторые коммуникационные агентства уже заявили, что оказывают подобные услуги, автоматически перенося термин в оф-лайн каждый раз, когда речь идет об интерактивных коммуникациях. IT-специалисты недовольно фыркают. Так, коммерческий директор студии «Интегрейт» Александр Ананченко категорически против подобного расширения термина. По его мнению, корректнее использовать (раз уж такая необходимость появилась) термин PR 2.0. 

\subsection{Каталоги сайтов и рубрикаторы}
\par Каталоги и рубрикаторы сайтов представляют собой систематизированную подборку ссылок на интернет-сайты с описаниями. Каталоги делятся на каталоги общего назначения и тематические (специализированные, отраслевые), а также на региональные, национальные и глобальные.1
\par В отличие от поисковых машин, где индекс создается автоматически, составление каталога – это ручная работа по классификации и аннотированию ресурсов. Ими удобно пользоваться в тех случаях, когда посетители ищут не ответ на конкретный вопрос, а сайты на определенную тему. Внутри представленных в каталогах категорий сайты могут быть отсортированы различными способами: по алфавиту, времени добавления ссылки, посещаемости или цитируемости.
\par Отдельного разговора заслуживает каталог Яндекса (yaca.yandex.ru). Он представляет собой описания сайтов русскоязычного Интернета, систематизированных по тематическим категориям. Сайты подбирает и описывает редакторская служба – редакторы отбирают самые качественные, по их мнению, ресурсы, которые должны помочь пользователям ориентироваться в интересующей теме. Благодаря каталогу Яндекса можно найти нужный сайт, сужая область поиска по теме и региону. В дополнение к обычной рубрикации по темам («Бизнес и экономика», «Дом и семья», «Развлечения и отдых» и пр.) администрация каталогов предлагает классификацию сайтов по типу информации (справки; товары и услуги; публикации и пр.). Кроме того, каталог является рейтингом описанных в нем ресурсов. Сайты в рубриках расположены по убыванию их тематического индекса цитирования (ТИЦ). 
\par Идея индекса цитирования проста. Как правило, люди, создающие сайт, ответственно относятся к его содержанию, а также к информации, на которую ссылаются, рекомендуя ее таким образом посетителям. Тематический индекс Яндекса учитывает количество ссылок на сайт с других ресурсов, придавая им разный «вес» (то есть значимость) в зависимости от авторитетности ссылающегося сайта. Как утверждается в описании каталога, в нем указаны десятки тысяч ресурсов. Ежемесячно с ним работают более 2 млн пользователей. Автоматическая система регулярно проверяет ссылки на предмет доступности и снимает с публикации те, которые перестали действовать. Регистрация в каталоге Яндекса – один из эффективных способов увеличить посещаемость сайта. Рассмотрение заявки может занять много времени, но в случае платной регистрации администрация каталога примет решение в течение одного рабочего дня.
\par Чем выше цитируемость сайта, тем эффективнее каталог Яндекса. Однако сайты даже с невысокой цитируемостью могут значительно увеличить свою посещаемость в первые дни присутствия в нем. Новые ресурсы в каталоге Яндекса указываются в первых строчках и привлекают повышенное внимание посетителей (рис. 13).
\par Подробнее о работе с каталогами сайтов можно узнать в разделах «Прописывание в каталогах общего назначения» и «Прописывание в тематических каталогах».

\subsection{Поисковые системы}
\par Поисковые системы (англ. search engine)  представляют собой инструменты для поиска информации в Сети. Они индексируют сайты с помощью специальных программ (поисковых роботов). Когда посетители поисковой системы задают тот или иной вопрос, она ранжирует проиндексированные сайты в соответствии с их релевантностью (см. раздел «Оптимизация под поисковые системы»).
\par Наиболее популярна в России поисковая система Яндекс (www.yandex.ru), ее доля в поисковом трафике составляет около 60 \%.1 Второе и третье места занимают Rambler и Google. Далее следует поиск Mail.ru (отметим, что в настоящее время эта система работает на технологиях Яндекса, и результаты двух поисковых систем совпадают). Замыкает пятерку MSN (msn.com). Находящаяся на шестом месте Апорт (www.aport.ru) дает не более 1 \%  поискового трафика.
\par Аудитории наиболее популярных в России поисковых систем несколько отличаются. Так, Rambler стал первой российской поисковой системой (стартовала в 1996 году), и многие посетители, которые начали работать с ней во второй половине 1990-х годов, до сих пор продолжают пользоваться именно ею.
\par Большинство тех первых пользователей теперь, спустя 10 лет, занимают руководящие посты в крупных и средних российских предприятиях, поэтому аудитория Rambler считается более «качественной» для корпоративного сектора. Данный тезис кажется наивным, однако многочисленные рекламные кампании, организованные автором, его подтверждали.
\par Есть особенности и у аудитории Google. Поскольку с ее помощью можно вести поиск по зарубежным англоязычным (и не только) источникам, она очень популярна среди IT-специалистов, научных работников и др.
\par Для упрощения одновременного поиска информации в нескольких поисковых системах созданы так называемые метапоисковые системы (их удобно использовать, например, занимаясь пресс-клиппингом). Их задача – предоставить пользователям информацию, полученную несколькими поисковыми системами. Собственной базы у них может и не быть, после запроса они обращаются к другим поисковым системам. В России наиболее известны метапоисковые системы Nigma  (www.nigma.ru) (рис. 15) и MetaBot  (www.metabot.ru).
\par Весьма перспективным представляется объединение рейтинга сайтов и поисковой системы. Подобные проекты могли бы ранжировать сайты, опираясь на пользовательские предпочтения. Так, если посетители, интересующиеся холодильникам и приходящие на некий сайт, сразу же покидают его, то его релевантность может быть уменьшена. Эта идея неоднократно обсуждалась специалистами, однако примеры успешной реализации в Рунете автору не известны.
\subsection{Системы управления контентом}
\par Сайты, предполагающие регулярное размещение информации, как правило, имеют систему управления контентом (англ. Content management system, CMS).  Благодаря ей можно минимально использовать помощь внешних технических специалистов.
\par Среди наиболее важных требований, предъявляемых к CMS,  выделим следующие.
\begin{itemize}
\item Возможность управления через администраторский ресурс структурой вывода страниц сайта без привлечения технических специалистов.
\item Возможность работы со значительными объемами информации без потерь в скорости.
\item Возможность поддержки неограниченного количества ресурсов в рамках одной лицензии.
\item Система должна выдерживать высокое нагрузочное тестирование.
\item Система должна быть гибкой и легко расширяемой.
\end{itemize}

\par Для большей части CMS  в качестве базового языка программирования используются Perl, Python  или PHP.  Все три языка – интерпретаторы (любые изменения ПО осуществляются быстрее и безболезненнее). Рассмотрим наиболее важные их особенности.
\par Perl  изначально разрабатывался для работы с большими массивами текста и начал использоваться в качестве средства разработки CMS,  когда еще не было других интерпретаторов под Unix.  Как правило, системы, написанные на Perl, работают качественно и быстро, однако вносить в них изменения крайне сложно, вследствие этого они крайне негибки и слабо расширяемы.
\par Python  – полноценный объектно-ориентированный язык высокого уровня. Системы, написанные на нем, отличаются относительно невысокой быстротой действия. Еще одним минусом использования данного языка является то, что в России работают всего лишь около 40–50 специалистов, способных писать CMS  на данном языке. В результате большая часть представленных в настоящее время на рынке CMS  работает в тестовом режиме (в России не существует ни одного коробочного продукта). При использовании мощного сервера система, написанная на Python,  выдерживает нагрузку около 5–6 тыс. посетителей в сутки.
\par PHP  – первый язык программирования, спроектированный специально для Интернета. В России достаточно много PHP-разработчиков, и большинство CMS написаны на нем. В основном используется модификация PHP4.  Однако около полутора лет назад появилась принципиально новая объектно-ориентированная модификация PHP5. 
\par PHP5 позволяет создавать приложения в режиме реального времени с использованием последних достижений IT-менеджмента экстремальное программирование). В то же время это полноценный объектно-ориентированный язык, что позволяет значительно увеличить скорость разработки за счет повторного использования кода. PHP5  обладает обновленными средствами контроля пользовательской активности, намного увеличивающими защищенность построенной на нем системы от действий недружественных пользователей.
\par В качестве баз данных обычно используются MySQL, PostgreSQL,  MSSQL и Oracle. 
\par MySQL  – является, по мнению автора данной книги, наихудшим среди перечисленных решений (из-за относительно низкой производительности в сложных интернет-проектах). За использование этой базой данных в отличие от PostgreSQL надо платить. Лицензия MySQL  не предполагает использования этой БД в коммерческих целях, однако, поскольку существует возможность установить MySQL свободно, многие провайдеры не обращают на лицензию внимания.
\par PostgreSQL – наилучшее решение для будущего сайта заказчика. На объемах данных больше чем 4 терабайта (1 терабайт = 1024 Гбайт) PostgreSQL сравним по производительности с MSSQL  и Oracle. 
MSSQL  автоматически подразумевает использование программного обеспечения Microsoft.  В результате резко увеличиваются аппаратные требования, возрастает стоимость сервера и программного обеспечения (использование лицензионного ПО обязательно).
\par База данных Oracle  – одно из лучших решений, но в рамках часто ее плюсы избыточны. Существенный выигрыш в производительности отмечается при работе с массивами данных больше 10 терабайт.
\par Таким образом, с точки зрения автора, лучшее решение для сложных интернет-ресурсов в России – это CMS, написанная на PHP5 и использующая в качестве базы данных PostgreSQL (рис. 17).
\par Еще одно требование к CMS  связано с необходимостью решить проблему работы сайта при больших нагрузках. Поскольку посещаемость ряда интернет-ресурсов может составлять тысячи и даже десятки тысяч посетителей в сутки, данная задача очень актуальна. В настоящее время есть два решения – использовать кэширующие механизмы или HTML-конвертер.
\par Кэширование предполагает запоминание сайта на некоторое время, тогда при обращении на сайт значительной части посетителей запросы в базу данных не отправляются. Однако из-за него сайт неадекватно отображается у некоторых пользователей (посетители могут видеть устаревший контент сайта). Еще один минус – увеличение требований к аппаратной части.
\par Второй вариант предполагает перманентную генерацию статических копий сайта с отдельными динамическими интерактивными вкраплениями в тех случаях, где это необходимо. В ответ на изменения на сайте перегенерируется именно та часть, где возникли изменения. HTML-конвертер не требует дополнительных ресурсных затрат.
\newpage

\chapter{Использование методов Интернет – маркетинга в корпорации Intel}
\section{Общая информация о корпорации Intel}
\par Американская корпорация, производящая широкий спектр электронных устройств и компьютерных компонентов, включая микропроцессоры, наборы системной логики (чипсеты) и др. Штаб-квартира — в городе Санта-Клара, штат Калифорния, США.
\par Компанию основали Роберт Нойс и Гордон Мур 18 июля 1968 года после того, как ушли из компании Fairchild Semiconductor. Вскоре к ним присоединился Энди Гроув.
Компания Intel является сегодня крупнейшим производителем полупроводников в мире. Она изменила наш мир не меньше, чем это сделали Apple и Microsoft в свое время (а если говорить точнее, то они ничего бы не сделали без Intel). Ведь Intel изобрела микропроцессор – сердце современных компьютеров. В начале XXI века процессоры Intel были установлены более чем на 80\% компьютеров по всему миру. Сегодня Intel выпускает достаточно широкий спектр продукции, который не заканчивается на одних лишь процессорах. Так, компания производит материнские платы, флэш-память, концентраторы и маршрутизаторы, концептуальные ноутбуки и многое другое [5].
\par Корпорация Intel на протяжении многих лет известна во всем мире как один из лидеров в сфере высоких технологий. Чтобы идти в ногу со временем и в течение длительного времени возглавлять бурно развивающуюся индустрию, необходимо не только проводить научные исследования и уникальные конструкторские разработки, но также постоянно искать новые пути построения маркетинговых коммуникаций с конечными потребителями.
\section{Применение методов интернет-маркетинга в корпорации Intel}
\par Для Intel целевой аудиторией являются, во-первых, специалисты в области информационных технологий. При чем это не только IT-менеджеры, принимающие решения, но и IT-профессионалы в общем. Эта часть аудитории характеризуется достаточно высокой квалификацией, она постоянно работает с компьютерами и стремятся быть в курсе последних технологических разработок в силу не только личной любознательности, но и служебных обязанностей.
\par Вторую и большую часть целевой аудитории составляют продвинутые пользователи. Это активная публика, не всегда хорошо подкованная в техническом плане, но стремящаяся знать как можно больше о лучших новинках и получить личный опыт пользования ими.
\par При этом и первая, и вторая группа целевой аудитории отличается прагматичным подходом к информации и открыта к eё получению. Также, общим для всей целевой аудитории является постоянное интенсивное пользование компьютерами. Решение о переносе мeдийных компаний в онлайн было принято Intel больше 10 лет назад. Если у большинства компаний на долю интернета приходится от 5 до 10\% общего мeдийного бюджета, то Intel делает основной акцент именно на развитии интернет-сообщества.
\par Конечно, сделано это по вполне прагматичной причине. Как уже говорилось, интернет является основным источников информации для целевой аудитории Intel и самым прямым и коротким путём общения с нею.
Однако недостаточно просто найти нужный канал коммуникации. Используя то или иное медиа, приходится учитывать его особенности. Поиск в нем нужных и, главное, достоверных сведений --- задача непростая и часто не находящая своего решения.
\par Традиционная баннерная реклама, хотя и обладает рядом преимуществ по сравнению с «медиа прошлого века» (например, доступность статистики), тем не менее, имеет и множество недостатков. В целом надо признать, что баннерная реклама чаще всего не воспринимается аудиторией, чему в немалой степени способствует тот факт, что современные браузеры позволяют отменить eе загрузку на веб-страницах.
\par Поэтому для максимально эффективного использования 100\% медийного бюджета Intel следовало ориентироваться на другие каналы коммуникаций, обеспечиваемые интернетом. Итак, в онлайн-продвижении было решено сделать упор на том, что всемирная паутина является интерактивной средой, предоставляя компаниям уникальную возможность вступать в прямой диалог со своими потребителями.
\par Именно по этой причине в качестве наиболее эффективной альтернативы традиционной интeрнет-рeкламе стали рассматриваться социальные сети. На данный момент ими пользуется более 90\% всех посетителей Рунета, что, безусловно, делает эти ресурсы мощным инструментом маркетинга. Однако, работая с социальными медиа, необходимо учитывать региональную специфику их аудитории. Централизованное управление маркетинговыми проектами в данной области практически невозможно и, как минимум, неэффективно. Демографический профиль аудитории крупнейших международных социальных сетей значительно разнится от региона к региону. Сильнее выражены отличия в предпочтениях и поведении пользователей.
\par Поэтому успешные проекты в социальных сетях были придуманы и организованы именно в российском представительстве Intel.
\par В июне 2007 года в социальной сети LiveJournal было создано сообщество i\_future, посвящённое информационным технологиям и их влиянию на будущее. Блоковая основа LiveJournal стала оптимальной платформой для реализации маркетингового проекта Intel, направленного на взаимный обмен пользователей информацией.
\par Сегодня i\_future, объединяя более 8300 пользователей, занимает первую строчку среди технических сообществ в LiveJournal. У проекта достаточно продвинутая аудитория, активно интересующаяся многими темами не только в ЖЖ, но также в других проектах и сетях. Благодаря этому наиболее интересные темы блогеров i\_future часто обсуждаются за пределами самого сообщества. Как следствие, нередки попадания в независимые рейтинги самых обсуждаемых блогов Рунета.
\par Характерная черта i\_future --- широкий тематический охват. Это буквально все, что касается сегодняшних и завтрашних технологий и разработок, а далеко не только продукты и решения от самой Intel. Современная архитектура и градостроительство, экология и «зеленные технологии», дизайн и концептуальные прототипы, научные исследования и новости из лабораторий, экономика и, конечно, IT --- все это обсуждается в сообществе. Доля контента, генерируемого при участии Intel, не превышает 30\%. Также в сообществе публикуется информация от партнёров, но основная часть контента создаётся в настоящее время уже усилиями самих блогеров --- участников сообщества.
\par Однако самым массовым проектом Intel в социальных сетях стала группа «Галактика Intel», появившаяся в сети «ВКонтакте» в ноябре 2008 года. Сегодня она насчитывает порядка 70 тыс. участников и является крупнейшей hardware-группой этой сети. Аудитория «ВКонтакте» очень сильно отличается от той, которая собрана компанией в сообществе i\_future. Она значительно моложе и слабее подготовлена в техническом плане, что не мешает ей активно интересоваться компьютерными новинками. Разумеется, с точки зрения маркетинга, аудитория, охватываемая сетью «ВКонтакте», также очень важна для Intel.
\par Участники группы интенсивно обсуждают специфические для eе состава темы, советуются друг с другом. Предоставление им информации в основном ведётся через комментируемую новостную ленту. Здесь также присутствует много партнёрской информации, проводятся конкурсы с призами.
\par Два проекта в области социальных сетей Intel развернула весной 2008 года непосредственно на своём российском сайте --- это «Галактика Intel» (Intel Galaxy) и IT Gаlаxy. Первый ориентирован на более широкую потребительскую аудиторию и связан с одноимённой группой «ВКонтакте». Второй является сообществом IT-профессионалов и органично дополняет аудиторию остальных социальных сетей.
\par Ключевая особенность аудитории проектов Intel Galaxy и IT Galaxy --- высокая степень доверия к бренду. На этих площадках компания периодически проводит различные активности в рамках программы лояльности, такие, например, как игра IT Manager III, позволяющая зарабатывать баллы, которые затем можно обменять на ценные призы. Эта игра представляет собой симулятор работы IT-специалиста. Сначала он управляет IT-отделом в небольшой компании, решает проблемы с техникой, обучает персонал. По мере развития компании увеличивается количество задач, растёт численность его отдела, появляются удалённые офисы. Новые проблемы приходится решать быстро, в рамках бюджета. Причём достичь наилучшего результата в игре позволяет применение технологий Intel. Основная задача игры: демонстрация возможностей современных технологий Intel в корпоративной игровой среде, максимально приближенной к реальной жизни. Например, последние обновление игры включает в себя сценарий «всемирного экономического кризиса», когда бюджет IT-отдела резко сокращается и его руководителю приходится решать вопрос: обосновать руководству необходимость получения дополнительного финансирования или работать в имеющихся условиях. Стоит отметить тот факт, что игроки из России входят в ТОП-20 мирового рейтинга (общее количество игроков в мире: более 58000, в России: более 8000).
\par При ориентации на совершенно разную аудиторию, Intel Galaxy и IT Galaxy имеют схожую структуру, основными компонентами которой служат:
\begin{itemize}
\item активно комментируемый новостной блог;
\item традиционный по формату форум;
\item конкурсы;
\item опросы.
\end{itemize}
\par На рисунке ~\ref{diagram1} показана диаграмма с процентным соотношением пользователей в описанных выше социальных проектах. Противовес в сторону группы Intel Galaxy понятен, в виду ориентации на массового пользователя.
\par Есть и специальные проекты, такие как «Клуб экспертов» на Intel Galaxy или «Лаборатория Intel» на IT Galaxy, которые дают членам сообществ различные привилегии. Например, «Лаборатория Intel» предоставляет возможность IT-профессионалам протестировать самое новое оборудование на базе технологий Intel. Техника установлена в лаборатории Intel в Нижнем Новгороде, любой желающий может записаться на консультацию, в рамках которой ему будет предоставлен удалённый доступ с возможностью загрузки своего ПО для тестирования и консультации специалистов Intel. Также в сообществе IT Galaxy любой желающий может пройти IT Galaxy IQ Test[5]. В режиме онлайн на экране в формате чата появляются вопросы на общие темы, а также на знание продуктов и новостей компании Intel. Победителем становится зарегистрированный участник, давший правильный ответ быстрее всех. Лучшие игроки становятся лидерами своеобразного рейтинга «умников» и получают призовые баллы в рамках программы лояльности Intel.
\begin{figure}
    \includegraphics[width=\linewidth]{diagram1.png}
    \caption{Количество пользователей в разных сообществах корпорации Intel}
	\label{diagram1}
\end{figure}
\section{Анализ SMM деятельности корпорации Intel}
\par Аудитория сообществ, сформировавшихся на данный момент вокруг сайта www.intel.ru, довольно велика (на IT Galaxy, например, зарегистрировано свыше 34 тыс. человек), что дат возможность использовать эти проекты в качестве площадки для экспериментов. После удачной обкатки в максимально лояльных и в то же время требовательных сообществах, новшества переносятся в другие проекты.
\par Уникальность проектов Intel Galaxy и IT Galaxy состоит также в предоставлении участникам этих сообществ возможности общаться с сотрудниками компании в оперативном онлайновом режиме. Сегодня постоянные блоги в этих проектах ведут более 10 инженеров и специалистов Intel. Получение информации «из первых рук», ответы на вопросы и участие в обсуждениях высококвалифицированных профессионалов – это действительно важные достоинства социальных сетей Intel.
\par В отличие от проектов в сетях «ВКонтакте» и LiveJournal, Intel Galaxy и IT Galaxy в большей степени концентрируются на технологиях и продукции Intel. Благодаря этим площадкам компании удалось сформировать своего рода базу знаний по этим вопросам, открытую для обсуждений, динамично пополняющуюся новыми публикациями, отслеживающую реальные интересы и потребности аудитории.
\par Привлечение новых участников в сообщества Intel Galaxy и IT Galaxy идет по нескольким основным направлениям:
\begin{itemize}
\item Через поисковые системы Рунета. При запросе, релевантном содержанию этих проектов, пользователь может обнаружить ссылки на Intel Galaxy или IT Galaxy в результатах поисковой выдачи и, перейдя по ним, оценить, насколько ему интересен такой контент.
\item Как уже говорилось, на эти сообщества идет прямая ссылка с корпоративного сайта компании www.intel.ru. Таким образом, любой заинтересованный посетитель сайта может стать участником сообщества.
\item Во время проведения различных конкурсов и викторин на крупнейших IT-площадках Рунета едет анонсирование этих мероприятий. Желающий принять участие в розыгрыше может перейти на сайт и зарегистрироваться в сообществе.
\item Также многие представители целевой аудитории регистрируются в сообществе по приглашению своих друзей и знакомых, уже являющихся членами IT Galaxy. Всем участникам, друзья которых зарегистрировались в сообществе и стали eе активными членами, начисляются призовые баллы в рамках программы лояльности.
\end{itemize}

\par Правом вносить свои темы для обсуждения в сообществе обладают все зарегистрированные пользователи. Здесь также работает система поощрений: авторы самых интересных материалов и полезных комментариев получают «баллы» в рамках программы лояльности Intel, которые затем могут обменивать на призы.
\par Обратившись к социальным сетям, Intel к настоящему времени построила надёжно работающий канале для двусторонних коммуникаций с конечными потребителями своих технологий и продуктов. Благодаря использованию разных по демографии и профессиональному составу сообщества этот эффективный инструмент маркетинга охватывает все важные для компании части целевой аудитории — от рядовых потребителей до IT-менеджеров, принимающих решения.
\par Интернет, независимо от внешних условий, развивается динамично. Появляются новые способы коммуникаций, открывающие принципиально новые, ранее неизвестные медийные возможности [6, с. 187]. Поэтому для успешного маркетинга в интернете любой компании нужно постоянно держать руку на пульсе событий, проявляя креативность и не боясь выйти за пределы традиционных подходов.
\par Результаты проделанной работы корпорации Intel по привлечению клиентов, используя интернет-маркетинг приведены в таблице ~\ref{money}. В ней показана валовая прибыль корпорации, полученная использованием интернет-маркетинга.
\begin{longtable}{|c|c|c|c|c|}
    \caption{Валовая прибыль от интернет-маркетинга корпорации Intel}\\ \hline
    \label{money} 
    Год & 2014 & 2015 & 2016 & 2017 \\ \hline
    Валовая прибыль, млн. \$ & 2,006 & 2,049 & 2,158 & 2,104 \\ \hline
\end{longtable}

\chapter{Применение системы Google Analytics в Intel}
\section{Характеристика программы Google Analytics}
\par Сайт в интернете представляет собой инструмент для продвижения определенного товара или вида услуг: будь то конкретный товар или ваши мысли о том, как его лучше продать.
\par Конечно же, как и любой продукт, сайт требует конкретной оценки своей эффективности.
С этой целью более 6 лет назад компанией Urchin Software было разработано программное обеспечение для анализа статистики посещения и трафика сайтов в сети интернет.
\par В апреле 2005 года компания Google выкупила Urchin Software Corporatoins, чтобы использовать этот продукт для определения эффективности своих рекламных проектов. В результате внедрения новых программ и настроек, Urchin был усовершенствован и в 2006 году получил название Google Analytics. Начиная с августа 2006 года, сервис Google Analytics стал доступен для всех желающих. И сегодня, по оценке агентства инновационного маркетинга Promo Interactive, Google Analytics является наиболее популярным инструментом статистики взаимодействия интернет-пространства и его пользователей Analytics (сокращённо GA) - бесплатный сервис, предоставляемый компанией Google для создания детальной статистики посетителей веб-сайтов. Статистика собирается на сервере Google, пользователь только размещает JS-код на страницах своего сайта.
\par Бесплатная версия ограничена 5 миллионами просмотров страниц в месяц. Пользователям с активным аккаунтом Google Ad Words предоставляется возможность отслеживания неограниченного числа просмотров страниц. Особенностью сервиса является то, что вебмастер может оптимизировать рекламные и маркетинговые кампании Google AdWords при помощи анализа данных, полученных при помощи сервиса Google Analytics, о том откуда приходят посетители, как долго они остаются на сайте и где они находятся географически.
\par Сервис интегрирован с Google AdWords. Пользователи видят группы объявлений и отдачу от ключевых слов в отчётах. Также доступны дополнительные возможности включая разделение посетителей на группы.
\par Пользователи сервиса могут определить цели и последовательности переходов. Целью может выступать страница завершения продаж, показ определённых страниц, или загрузка файлов. Используя данный инструмент, маркетологи могут определять, какая из рекламных кампаний является успешной, и находить новые источники целевой аудитории.
\par Любой пользователь может добавить до 50 профилей сайтов. Каждый профиль обычно соответствует одному сайту Analytics показывает основную информацию "на панели инструментов", более детальную информацию можно получить в виде отчёта. В данный момент доступно 80 видов настраиваемых отчётов.
\par Благодаря настройке RSS, можно отслеживать переходы по соответствующей ссылке. При этом расположение ссылок не важно - это могут быть тексты, реклама и даже картинки. Важно знать, как анализировать информацию по целям. Для этого нужно обратить внимание на левое меню. Специалисты советуют, что если кликов было 300, эту цифру следует делить на два, так как кто-то кликнул по ошибке, а кто-то из любопытства, то есть фактическое количество кликов уменьшается. Analytics представляет собой бесплатный сервис, с помощью которого можно получить детальную статистику посетителей любого сайта. Для этого достаточно на каждой странице сайта установить соответствующий код JavaScript (JS-код). У любого пользователя есть возможность добавить до 50 профилей сайтов, и в бесплатном варианте осуществлять до 5 миллионов просмотров страниц в месяц.
\par Статистика Google Analytics позволяет выяснить, с какой страницы интернета пришли посетители на ваш сайт, где они расположены в это время географически и сколько времени они находились на вашем сайте.
\par Данные Google Analytics эффективно используются оптимизаторами для законной раскрутки сайтов. Умелое их применение способствует продвижению сайта и достижению результатов показов на первых страницах поисковых систем.
\par Помимо основной информации, при помощи дополнительных настроек в Google Analytics можно получить более 80 видов отчетов, например, таких как статистика переходов по ссылкам и анализ поведения пользователей при работе с сайтом.
\begin{figure}
    \includegraphics[width=\linewidth]{ris1.png}
    \caption{Показатели Google Analytics}
	\label{ris1}
\end{figure}
\par При анализе статистики Google Analytics важно четко различать основные его показатели:
\begin{itemize}
\item Уникальные посетители - это неповторяющиеся (учитываемые только один раз) посетители Вашего сайта за указанный период времени. Уникальным считается посетитель, зашедший на ваш сайт с одного IP-адреса.
\item Сеанс посетителя - это период взаимодействия между браузером посетителя или вашим сайтом, который завершается при закрытии окна браузера, завершении работы браузера или неактивности пользователя на сайте на протяжении 30 минут.
\item Сеансы возврата - это количество возвращений на Ваш сайт уникальных посетителей за указанный период времени.
\end{itemize}
\par Уникальные посетители формируют рейтинг вашего сайта, и, помимо проявления заинтересованности в предлагаемом вами продукте, делают сайт привлекательным для рекламодателей.
\par Итак, в первую очередь нужно определиться, что можно делать при помощи этой программы.
\par Значительные преимущества Google Analytics:
\begin{itemize}
\item можно отслеживать статистику переходов на сайт;
\item классифицировать посетителей (такой подход позволяет разрабатывать новые страницы сайта целенаправленно);
\item можно отслеживать исходящие ссылки, используемые при продвижение сайта, что также необходимо для дальнейшей раскрутки сайта;
\item можно отслеживать ссылки, которые скачивают больше всего;
\item можно отследить адреса электронной почты по кликам;
\item практически все коммерческие транзакции прослеживаются при помощи этого программного обеспечения;
\item можно также отключить статистику посещений тех, кто обслуживает сайт;
\item можно сравнить статистики посещения сайта за разные периоды (данная возможность идеальна для выявления эффективности работы новых страниц).
\end{itemize}
\par Недостатки у Google Analytics все же есть:
\begin{itemize}
\item отследить трафик в случае, если у пользователя отключены cookies файлы невозможно;
\item кроме того, Google Analytics не может повторно обработать данные, если потерян профиль с настроенными фильтрами;
\item настройка отчетов имеет ограниченное количество (в частности можно настроить до 12 разных отчетов);
\item количество отслеживаемых целей также ограничено (в настоящее время Google.Analytics отслеживает до четырех целей. К слову, каждая цель может быть конвертирована для получения дополнительной прибыли).
\end{itemize}
\section{Краткое описание работы в Google Analytics}
\par Для регистрации нового аккаунта или входа в систему, нужно зайти на официальный сайт Google: www.google.com/intl/ru/analytics/ далее действовать согласно инструкциям.
\par Если уже есть аккаунт в любом из сервисов Google (GMail, AdWords, Adsense и т.п.), то регистрироваться заново не обязательно. Только если не хотите иметь отдельный аккаунт для Google Analytics. Не будем подробно останавливаться на процессе регистрации, потому что там все просто.
\par Настройка профиля сайта (рис.3.2.) в аккаунте Google Analytics: чтобы данные обрабатывались правильно, надо настроить профиль сайта. В текущем окне, справа от названия профиля, нажать на ссылку "Изменить", и откроются настройки этого профиля:
\begin{figure}
    \includegraphics[width=\linewidth]{ris2.png}
    \caption{Настройка профиля Google Analytics}
	\label{ris2}
\end{figure}
\par Здесь главное правильно указать "Страницу по умолчанию" - это главная страница сайта (обычно index.php или index.html). И правильный часовой пояс. Оба этих параметра влияют на данные в отчетах.
Рассмотрим еще одну страницу- это "Панель инструментов" (Рисунок ~\ref{ris3}).
\begin{figure}
    \includegraphics[width=\linewidth]{ris3.png}
    \caption{Панель инструментов Google Analytics}
	\label{ris3}
\end{figure}
\par Данная страничка представляет собой "краткое содержание" системы, по умолчанию это:
\begin{itemize}
\item График посещений
\item Использование сайта
\item Обзор посетителей
\item Наложение данных на карту
\item Обзор источников трафика
\item Обзор содержания
\end{itemize}
\par Если что-то не нужно на этой панели - вы можете удалить отчет, или добавить другой, можете поменять расположение элементов, в общем удобно.
\par Глядя на график можно заметить, что первые посетители пришли на сайт 5-го мая. Подобный график будет "преследовать" практически на каждой странице статистики. Далее рассмотрим страницу "Обзор посетителей" (Рисунок ~\ref{ris4})
\begin{figure}
    \includegraphics[width=\linewidth]{ris4.png}
    \caption{Обзор посетителей Google Analytics}
	\label{ris4}
\end{figure}
\par Первая страничка раздела "Посетители" предоставляет общую информацию без ненужной детализации.
\par Источники трафика --- это раздел для тех кто занимается раскруткой сайтов, первая страничка "Обзор" соответствует своему названию:
\begin{figure}
    \includegraphics[width=\linewidth]{ris5.png}
    \caption{Источники трафика Google Analytics}
	\label{ris5}
\end{figure}
\par Далее есть много подпунктов данного раздела:
\begin{itemize}
\item Прямой трафик --- статистика пришедших по URL введенному вручную в браузере (или тех кому отправили ссылку по какому-либо мессенджеру).
\item Сайты-источники переходов --- оставили ссылку на сайт где-то на просторах Интернета - вот и пришли гости.
\item Поисковые системы --- кто-то что-то искал, а нашёл именно этот сайт.
\item Все источники трафика --- результирующая страничка по предыдущим трём.
\item Ключевые слова --- когда ктото ищет этот сайт в поисковиках и по какому слову нашел этот сайт. Данный пункт предназначен для работы с AdWords.
\item Кампании --- тут перечислены кампании AdWords, почему не подпунктом AdWords.
\item Версии объявлений --- работа с объявлениями AdWords.
\end{itemize}
\par Содержание - еще один не маловажный пункт статистики, можно просмотреть какие разделы сайта наиболее интересны пользователям, и какие наиболее популярные точки выхода (Рисунок ~\ref{ris6}).
\begin{figure}
    \includegraphics[width=\linewidth]{ris6.png}
    \caption{Обновление данных "Google Analytics"}
	\label{ris6}
\end{figure}
\section{Эффективность внедрения программы Google Analytics на примере Intel}
\par Проведение анализа поведения пользователей на сайте - необходимая часть внедрения проекта в Интернете. И чем проект сложнее и разветвленнее, чем крупнее.
\par С помощью Google Analytics грамотный специалист сможет определить основные аспекты поведения посетителя на сайте и его seo блог всегда будет посещаем и доступен в первых рядах поисковой выдачи. Он будет знать сколько времени пользователь провел на сайте, какие страницы посещал, и с каких он покинул сайт.
\par Однако, это не значит, что начинающему блогеру не нужно заглядывать и разбираться с данными аналитики.
\par Результатом работы Google Analytics является предоставление файла с отчетом, в котором представлена вся информация о посещаемости, сделаны выводы, и дальнейшие рекомендации по улучшению сайта. Скачивайте данные отчеты себе на компьютер, и периодически возвращайтесь, сравнивайте с новыми значениями, обязательно отмечая проделанную работу и какие она повлекла результаты
\par Компании Intel удалось оптимизировать свой сайт на основании отчетных данных Google Analytics с помощью функции наложения данных на сайт Google Analytics и отчетов, показывающих, что компания теряет почти половину клиентов. Компания сократила процесс до одного шага и свели его на одну страницу, поэтому количество посетителей увеличилось на 50\%. По прогнозам, это будет способствовать значительному росту доходов в ближайшие несколько месяцев. Продолжает тестировать и контролировать программы маркетинга в интернете с помощью веб-аналитики. До Google Analytics инетрнет-компания тратила маркетинговый бюджет, главным образом, наугад. Теперь, зная, в какой мере кампании окупаются и насколько они эффективны, система Google Analytics оказала большое положительное воздействие на бизнес фирмы.
\par Возьмем в качестве примера показатели статистики сайта Intel. Согласно данным, представленным Google Analytics, за период с 1 мая 2015 года по 1 мая 2016 года, сайт посетили 2 202 000 посетителей, из которых 1 066 000 - уникальных.
\par Физическое местонахождение посетителей сайта Inte; зафиксировано в 37 странах мира, из которых: 81\% процента посетителей сайта из Америки, Китай - 9\%, остальная часть распределена между странами Европы и Средней Азии. Сравнивая эти показатели, можно выявить процент популярности сайта на определенной территории, в данном случае - в Америке.
\par Точно также, пользуясь статистикой Google Analytics, а также необходимыми для сравнения данными, можно определить для своего сайта: территориальное размещение заинтересованных посетителей, процент посетителей вашего сайта в общем количестве пользователей и другие важные показатели.
\par При этом важно иметь в виду, что количество посещений отражает не только потребность в вашем продукте, но и эффективность работы вспомогательных служб вашей компании - маркетологов и веб-мастеров.

\starchapter{Заключение}
\par Коммерческое использование Интернета, в значительной степени связанное с появление и развитием службы World Wide Web, насчитывает менее чем одно десятилетие, однако за этот небольшой промежуток времени произошло громаднейшее число самых разнообразных событий, рождение большого числа новых компаний. Обороты рынка электронной коммерции за это время выросли во много раз и скоро достигнут отметки в \$ трлн. Компаниям Интернет предоставил новый инструмент ведения бизнеса, средство снижения издержек и более полного удовлетворения потребностей потребителей. Потребители, в свою очередь, получили новый информационный источник о товарах и услугах, новые пути удовлетворения своих потребностей за счет возможности взаимодействия с более широких кругом компаний и новое эффективное средство коммуникации, как с компаниями, так и между собой.
\par Этот период зарождения электронного бизнеса выявил два важных момента. Во-первых, Интернет доказал свою высокую эффективность, как средства коммуникации, и высокий потенциал построенного на его основе глобального электронного рынка. Во-вторых, опыт компаний, либо пытающихся использовать Интернет, как дополнение своего традиционного (off-line) бизнеса, либо изначально построивших свой бизнес в Интернете, подтвердил важность и необходимость учета и использования всего существующего опыта по ведению коммерческой деятельности и использованию принципов маркетинга в своей деятельности.
\par Наряду с бурным ростом электронного бизнеса одним из важных явлений стало появление нового направления в маркетинге - интернет-маркетинга. В некоторых источниках это направление также именуется как гипермаркетинг, в котором приставка гипер- подчеркивает гипермедийный характер среды Интернета. Все эти названия объединяет та сущность, которая лежит в основе глобальной компьютерной Сети - это гипер- и мультимедийная глобальная компьютерная среда предоставляющая невиданные до сих пор возможности взаимодействия, начиная от простого обмена информацией, кончая осуществлением финансовых транзакций, заключением сделок и доставкой цифровых продуктов.
\par В сегодняшней коммерческой практике весьма интенсивно используется Интернет. В ходе проведенного исследования установлено, что интенсивность использования интернет-технологий в предпринимательской практике возрастает. Большинство коммерческих структур, фирм организаций и учреждений с помощью технологий Интернета успешно осуществляют коммерческие операции. Интернет на современном этапе развития предпринимательства выступает как специфический элемент развития рыночной инфраструктуры. Система Интернета позволяет создавать и распределять информационные потоки, формировать бизнес-сообщества (интернет-компании). По сути Интернет создает предпосылки для формирования специфического сектора бизнеса.
\par Результаты работы над данным дипломным исследованием позволяет установить, что использование интернет-технологий создает предпринимателям несколько важных направлений повышения эффективности бизнеса: ускорение процесса платежных операций, повышение оперативности коммунальных связей, использование Интернета, как дополнительного канала информационных потоков.
\par Во-первых, на фоне бурно развивающегося и изменяющегося Интернета темпы преобразований в областях маркетинга и рекламы в Сети впечатляют. За последние годы и в мире, и в России Интернет-реклама стала заметным и самостоятельным бизнесом. Но в силу того, что российский Интернет далеко не во всем похож на Интернет Европы и США, и реклама в нашем Интернете существенно отличается от рекламы в иных странах, полноценное использование Интернета как маркетингового инструмента в России только начинает зарождаться.
\par Во-вторых, Интернет дает множество инструментов для воздействия на целевую аудиторию при проведении рекламных кампаний. Среди них можно выделить: размещение рекламы на тематических и общеинформационных сайтах, баннерные сети, e-mail маркетинг, продвижение с помощью поисковых систем и каталогов, обмен ссылками, рейтинги и т.д.
\par Но помимо приведенных способов интернет-рекламы, наличие самого сайта рекламодателя является обязательным условием при проведении рекламных кампаний.
\par Компания, имеющая свой сайт и занимающаяся рекламой в Интернете имеет ряд преимуществ перед традиционной рекламой: более низкая стоимость рекламной кампании по сравнению с традиционными СМИ; большая аудитория, чем у СМИ; возможность направления потока рекламы только на целевую аудиторию; возможность таргетинга рекламных кампаний в зависимости от портрета целевой аудитории; возможность оперативного управления рекламными кампаниями; возможность оценки эффективности рекламы более качественной, чем у традиционных СМИ.
\par В-третьих, интернет-реклама - это высоко технологичное средство коммуникаций, которое позволяет в ходе рекламной кампании собирать данные о количестве показов рекламных баннеров, отслеживать нажатия на баннеры, но самой удобной возможностью является то, что можно оперативно менять баннеры, которые имеют низкий отклик на более эффективные, тем самым корректирую рекламную кампанию в течение всего периода.
\par Наконец, исследование всех теоретических и прикладных аспектов позволяет разобрать методику анализа и оценки эффективности рекламных кампаний с использованием Интернета.
\par По завершению рекламной кампании подводятся итоги, собираются и анализируются данные, которые в последствии требуется изучить и получить информацию о том, на сколько эффективно прошла рекламная кампания, что следует учесть при следующих рекламных кампаниях, нужно ли продолжать рекламную кампанию вообще.
\par Данные для анализа предоставляются системами статистики, необходимо иметь доступ к статистике сайта или баннерной сети, где происходит реклама, а также иметь собственную статистику сайта.
\par Делая вывод - Google.Analytics многофункциональный сервис, который позволяет осуществлять раскрутку сайта более эффективно. Для этого важно научиться пользоваться панелями инструментов этого программного обеспечения. Конечно, минусы у Google.Analytics все же есть, но они при всех преимуществах незначительны. Именно поэтому, многие сегодня пользуются этой программой для отслеживания трафика на своем сайте, исключая неэффективные методы раскрутки в самом начале работы.

\newpage
\begin{thebibliography}{1}
	\bibitem{1} Амблер Т. Практический маркетинг / Т. Амблер. - СПб.: Питер, 1999. - 400 с.
\bibitem{2} Багрин Ю. Интернет как новый маркетинговый канал // Маркетинг и реклама. 2009. -- № 1. 2.
\bibitem{3} Бокарев Т.А. Способы продвижения компании в сети Интернет// Маркетинг и маркетинговые исследования в России, 2009. - № 4.
\bibitem{4} Бурдинский А.А. Интернет-маркетинг как новый инструмент развития бизнеса// Маркетинг и маркетинговые исследования в России, - 2005. - № 2.
\bibitem{5} Бушуева Л.И. Роль Интернет-услуг в практической маркетинговой деятельности // Маркетинг в России и за рубежом №4, 2001 г.
\bibitem{6} В. Холмогоров. Интернет-маркетинг. Краткий курс. - Питер, 2002 г., с: 272
\bibitem{7} Голик В.С. Эффективность Интернет-маркетинга в бизнесе. -Дикта, 2008 г., с: 196
\bibitem{8} Голубков Е.П. Использование Интернета в маркетинге // Маркетинг в России и за рубежом №3 (29), 2002 г.
\bibitem{9} Гуров В. Интернет для бизнеса. -М.:Дело, 2006. -289 с.
\bibitem{10} Данько Т.П. Управление Интернет-маркетингом: Учебное пособие.- М.: Инфра-М, 2007.
\bibitem{11} Закарян И., Филатов И. Интернет как инструмент интернет-маркетинга. -СПб.:BHV, 2006. - 302 с.
\bibitem{12} Интернет-маркетинг на 100 \%. -Питер, 2009 г., с: 240
\bibitem{13} Колесников О.Э. Интернет для делового человека. -М.:МЦФ. Издат. Фирма Яуза, 2004 -281 с.
\bibitem{14} Михайлова Е.А. Проблемы и перспективы взаиморазвития Интернета и международного маркетинга // Маркетинг в России и за рубежом, 2003. - №6.
\bibitem{15} Михаил Зуев, Денис Разваляев. Интернет-маркетинг. Взгляд практиков. - Вершина, 2008 г., с: 248
\bibitem{16} Использование интернет-технологий в маркетинговой деятельности- М.: Омега Л,2007
\bibitem{17} Пол Гринберг. CRM со скоростью света. Привлечение и удержание клиентов в реальном времени через Интернет. - Символ-Плюс, 2006 г., с: 530
\bibitem{18} Пименов Ю.С. Использование Интернет в системе маркетинга//Маркетинг в России и за рубежом. -2003. -№1. -с. 2-5.
\bibitem{19} Успенский И.В. Интернет-маркетинг Учебник.- СПб.: Изд-во СПГУЭиФ, 2003 г.
\bibitem{20} Филипп Гуров. Продвижение бизнеса в Интернет. Все о PR и рекламе в Сети. - Вершина, 2008 г., с: 152
\end{thebibliography}
\end{document}
